\RequirePackage[l2tabu, orthodox]{nag}
% Autor: Kamil Ziemian
\documentclass[a4paper,11pt]{article}
\usepackage[utf8]{inputenc}
\usepackage[polish]{babel}
\usepackage[MeX]{polski}
\usepackage{microtype}
\usepackage{xcolor}
\usepackage{vmargin}
% --------------------------------------------------------------------
% MARGINS
% --------------------------------------------------------------------
\setmarginsrb
{ 0.7in}  % left margin
{ 0.6in}  % top margin
{ 0.7in}  % right margin
{ 0.8in}  % bottom margin
{  20pt}  % head height
{0.25in}  % head sep
{   9pt}  % foot height
{ 0.3in}  % foot sep
\usepackage{graphicx}
\usepackage{hyperref}
% ##############################


% ##################
\renewcommand{\arraystretch}{1.2}


\newcommand{\spaceOne}{3em}
\newcommand{\spaceTwo}{2em}
\newcommand{\spaceThree}{1em}
\newcommand{\spaceFour}{0.5em}


\newcommand{\tb}{\textbf}
\newcommand{\noi}{\noindent}
\newcommand{\start}{\noi \tb{--} {}}
\newcommand{\Center}[1]{\begin{center} #1 \end{center}}
\newcommand{\CenterTB}[1]{\Center{\tb{#1}}}
\newcommand{\Str}[1]{\tb{Str. #1.}}
\newcommand{\StrWg}[2]{\tb{Str. #1, wiersz #2.}}
\newcommand{\StrWd}[2]{\tb{Str. #1, wiersz #2 (od dołu).}}
\newcommand{\Jest}{\tb{Jest: }}
\newcommand{\Pow}{\tb{Powinno być: }}
\newcommand{\Prze}{{\color{red} Przemyśl.}}
\newcommand{\Dok}{{\color{red} Dokończ.}}
\newcommand{\Main}[1]{ \begin{center} {\huge \tb{#1} } \end{center} }
\newcommand{\Field}[1]{ \begin{center} {\LARGE \tb{#1} } \end{center} }
\newcommand{\Work}[1]{ \begin{center} {\large \tb{#1}} \end{center} }

\newcommand{\red}[1]{{\color{red} #1}}





% ####################################################################
\begin{document}
% ####################################################################



% ########################################
\Main{Historia, błędy i~uwagi}

\vspace{\spaceTwo} \vspace{\spaceThree}
% ########################################



% ##############################
\Field{Historia świętej wiary}

\vspace{\spaceTwo} % \vspace{\spaceThree}
% ##############################



% ####################
\Work{
  Richard Butterwick \\
  ,,Polska Rewolucja a~Kościół Katolicki 1788--1792'',
  \cite{ButterwickPolskaRewolucjaAKosciolKatolicki12} }


\CenterTB{Uwagi}

\start \Str{28} Euzebiusz \\


% \CenterTB{Błędy}
% \begin{center}
%   \begin{tabular}{|c|c|c|c|c|}
%     \hline
%     & \multicolumn{2}{c|}{} & & \\
%     Strona & \multicolumn{2}{c|}{Wiersz}& Jest & Powinno być \\ \cline{2-3}
%     & Od góry & Od dołu &  &  \\ \hline
%     & & & & \\
%     & & & & \\
%     & & & & \\
%     & & & & \\
%     %     \hline
%   \end{tabular}


\vspace{\spaceTwo}





% ####################
\Work{
  Warren H.~Carroll \\
  ,,Historia Chrześcijaństwa. Tom~IV: Podział Chrześcijaństwa'',
  \cite{CarrollHistoriaChrzecijanstwaTomIV11} }


\CenterTB{Błędy}
\begin{center}
  \begin{tabular}{|c|c|c|c|c|}
    \hline
    & \multicolumn{2}{c|}{} & & \\
    Strona & \multicolumn{2}{c|}{Wiersz}& Jest & Powinno być \\ \cline{2-3}
    & Od góry & Od dołu &  &  \\ \hline
    % & & & & \\
    % & & & & \\
    % & & & & \\
    % & & & & \\
    % & & & & \\
    % & & & & \\
    % & & & & \\
    % & & & & \\
    % & & & & \\
    % & & & & \\
    % & & & & \\
    % & & & & \\
    % & & & & \\
    % & & & & \\
    % & & & & \\
    % & & & & \\
    % & & & & \\
    % & & & & \\
    % & & & & \\
    % & & & & \\
    % & & & & \\
    817 & 12 & & Hilaire. & Hilaire, \\
    817 & & 7 & Henrich. & Henrich, \\
    817 & & 2 & Anthony. & Anthony, \\
    818 & & 4 & E.H. & E.H., \\
    818 & & 1 & Philippe. & Philippe, \\
    819 & & 9 & 1913 & 1913. \\
    820 & & 1 & 1992.. & 1992. \\
    823 & 2 & & \emph{1621--9} & \emph{1621--1629} \\
    825 & 5 & & Charles. & Charles, \\
    825 & 11 & & John., & John, \\
    825 & & 5 & Carlos. & Carlos, \\
    825 & & 2 & Francis. & Francis, \\
    825 & & 1 & Francis. & Francis, \\
    % 825 & & & & \\
    826 & & 8 & \emph{leyasu} & \emph{Ieyasu} \\
    826 & & 4 & R.S. & R.S., \\
    \hline
  \end{tabular}
\end{center}

\vspace{\spaceTwo}





% ####################
\Work{
  Warren H.~Carroll, Anne W. Carroll \\
  ,,Historia Chrześcijaństwa. Tom~VI: Kryzys Chrześcijaństwa'',
  \cite{CarrollHistoriaChrzecijanstwaTomVI14} }


\CenterTB{Uwagi}

\start \Str{12} Wcięcia wszystkich akapitów poza pierwszy~są zbyt
duże.

\vspace{\spaceFour}


\start \StrWd{31}{4--2} Zdanie ,,Jestem zobowiązany Jamesowi
H.~Billingtonowi, \emph{Fire In the~Minds~of Man}, wielkiemu
historykowi myśli rewolucyjnej'' po polsku brzmi źle i~jest trochę bez
sensu. Nie wiem jednak jak je~poprawić.

\vspace{\spaceFour}


\start \Str{33} Jest dziwne, że~Lamennais jest tu nazwany ,,wielkim,
choć czasami błądzącym, francuskim duchownym'', skoro sama ta książka
podaje na~43 stronie, że~odrzuci on najpierw wiarę katolicką, potem
zaś chrześcijaństwo. Możliwe, że~ta nielogiczność jest wyniki
pośmiertnej edycji i~uzupełniania tego dzieła oraz pracy tłumacza.

\vspace{\spaceFour}


\start \Str{54} Pisze tu, że~bitwa pod Nowym Orleanem była decydującym
momentem w~Wojnie~1812 roku, powołując~się na książkę Paula Johnsona
\emph{Birth~of the~Modern}. Jednak w~tej pozycji Johnson przedstawia
zupełnie inną wersję wydarzeń. Bitwa ta rozegrała~się już po zawarciu
pokoju w~Londynie \red{Sprawdź miasto}, ale~przed tym jak statek
z~informacją o~tym dotarła do~USA, jej przebieg nie doprowadził jednak
do~kontynuacji działań wojennych. Tym samym, konkluduje Johnson, nie
wpłynęła na zawarcie pokój, ale~bardzo na~jego recepcję. Amerykanie
mogli~się bowiem czuć zwycięzcami wojny jako, że~wygrali ostatnią jej
bitwę.

\vspace{\spaceFour}


\start \Str{63} Możliwe, że~informacje podane na tej i~na następnych
stronach dotyczące Ameryki Łacińskiej są poprawne, jednak napisane są
w~sposób pełen luk i~niejasności. Na~przykład na dole tej strony jest
podane, że~Martin skapitulował przed Monteverdim i~wyjechał
z~Wenezueli, zaraz potem zaś~został zdradzony, aresztowany i~wysłany
przez Bolivara do~Hiszpanii w~zamian za paszport, który umożliwi mu
przyjazd do~Starego Kraju. Wydaje~się mało prawdopodobne, by~Bolivar
mógł aresztować Martina, gdyby ten opuścił już Wenezuelę.

Poza tym, nie ma żadnego jasnego stwierdzenia, że~Bolivar wykorzystał
paszport i~udał~się do~Hiszpanii. Zaraz po~informacji, że~zdobył ten
dokument przenosimy~się do Trujillo dnia 15~czerwca 1813, co może
oznaczać miasto w~Hiszpania, ale~też jedno z~wielu o~takiej nazwie
w~Ameryce Południowej. Pierwszym pewnym miejsce w~którym go potem
widzimy, jest wenezuelska Barcelona.

\vspace{\spaceFour}


\start \StrWd{67}{8} Po~tej linii powinien nastąpić odstęp między
przypisami.

\vspace{\spaceFour}


\start \Str{76} Następcą zmarłego w~1820~roku Jerzego~III
Hanowerskiego był jego najstarszy syn Jerzy~IV Hanowerski panujący
w~latach 1820--1830. Dopiero po~nim panował w~latach 1830--1837
panował Wilhelm~IV, który był młodszym synem Jerzego~III, a~nie jego
dalekim krewnym. Z~tego tej karygodnej pomyłki wszelkie dalsze
odniesienia do~działań tego monarchy mogą być błędnie przypisanymi mu
aktami Jerzego~IV, bądź źle umieszczone w~czasie.

\vspace{\spaceFour}


\start \StrWd{83}{20--17} Zdanie ,,Tak samo było w~przypadku Lenina,
kolejnego wielkiego przywódcy rewolucji, który wychował~się w~pobożnej
chrześcijańskiej rodzinie, a~fakt, że~wedle jego własnego świadectwa,
utracił wiarę w~wieku szesnastu lat, nie miał na~to żadnego wpływu.''
źle brzmi i~bardzo trudno zrozumieć myśl jaką w~tym kontekście miało
przekazywać.

\vspace{\spaceFour}


\start \Str{110} Na~tej stronie jest podane, że~gdy~w~1914 roku
zamordowano arcyksięcia Franciszka Ferdynanda i~jego żonę Zofię,
Franciszkowi Józefowi wyrwał~się raz jedyny okrzyk ,,Nie oszczędzono
mi niczego!'', podczas gdy na~stronie~115 jest napisane, iż~wykrzyknął
on ,,Nie oszczędzono mi niczego na~tej ziemni'' w~momencie,
gdy~dowiedział~się o~zamordowaniu swojej żony Elżbiety. Te~dwa
fragmenty zdają~się sobie przeczyć.

\vspace{\spaceFour}

\start \Str{125} W~drugim paragrafie na~tej stronie jest trochę
zamieszani. Na~początku jest mowa o~zebraniu 87 osób szwajcarskim
Vevey. Na~samym jego końcu jest mowa o~głosowaniu w~kortezach i~ilości
głosów jaka tam padła, co~nie ma chyba nic wspólnego z~tym zebraniem
i~ilością osób która na nim była, nie~pamiętam zaś aby w~tej książce
była podana ilość osób zasiadających w~kortezach.

\vspace{\spaceFour}


\start \StrWd{126}{8} Nie wiem czemu w~tej linii umieszczono słowa
\emph{Dios! Patria! Fueros! Rey!}

\vspace{\spaceFour}


\start \Str{135} Fragment utworu poety Grillparzera o~marszałku
Radetzkim jest tu cytowany z~innego źródła niż na~następnej stronie.
Nie jest to żaden błąd, jedynie trochę to dziwne.

\vspace{\spaceFour}


\start \Str{145} Dwa ostatnie paragrafy nie~mają wcięcia w~tekście.

\vspace{\spaceFour}


\start \Str{147} Stwierdzenie, że~to święty Piotr ustanowił papiestwo
i~, ten błąd jest szczególnie karygodny, Kościół jest sprzeczne
z~wiarą katolicką. Zapewne jest to herezja, lecz nie jestem na tyle
kompetentny by~stwierdzić to na 100\%. Jeśli jest to herezja, to
wątpię by obarczała sumienie Carrolla, który zapewne po prostu
popełnił głupi błąd pisząc te słowa.

\vspace{\spaceFour}


\start \Str{151} Przynajmniej w~mojej opinii na~tej stronie panuje
pewne zamieszanie. Nie potrafię na~przykład z~całą pewnością
stwierdzić, które z~wydarzeń opisanych w~ostatnim paragrafie
odnoszą~się do~pierwszego synodu, a~które do drugiego.

\vspace{\spaceFour}


\start \StrWd{165}{14--12} Sens zdania ,,Wielu opuszczało ojczyznę,
wypływając do~USA z~niewielkich portów, a~ich nazwiska przetrwały
tylko w~lokalnej tradycji.'' jest następujący. Pamięć o~tym, kto
wówczas wypłynął do~Stanów Zjednoczonych zachowała~się w~lokalnej
tradycji ustnej, ale~nie w~dokumentach z~tamtej epoki. W~tym sensie
ich nazwiska nie przetrwały w~źródłach, nie należy jednak przez to
rozumieć, że~ich nazwiska zniknęły z~użycia, co taka forma tego zdania
może sugerować.

\vspace{\spaceFour}


\start \Str{173} Mam problem ze zrozumieniem opisanych tu powodów
wybuchu wojny francusko\dywiz pruskiej. Dlaczego niby informacja
o~tym, że~Niemcy obrażają Francuzów wysłana do~króla Prus Wilhelma
miała spowodować wypowiedzenie wojny przez Napoleona~III.

\vspace{\spaceFour}


\start \Str{218} Na~dole strony pozostawiono puste miejsce, które
powinien zajmować tekst z~następnej strony.

\vspace{\spaceFour}


\start \StrWd{225}{3} Po tej linii następuje za~duży odstęp.

\vspace{\spaceFour}


\start \Str{264} Dwa pierwsze paragrafy są źle sformatowane.

\vspace{\spaceFour}


\start \Str{274} Na~dole strony pozostawiono puste miejsce, które
powinien zajmować tekst z~następnej strony.

\vspace{\spaceFour}


\start \Str{277} Należy sprawdzić, czy w~czasie rebelii tajpingów nie
zginęło na~polach bitew więcej osób, niż podczas I~Wojny Światowej.
Uwaga którą tu poczynił Carroll\footnote{Myślę, że~Anne W.~Carroll
  zgodziłaby~się na~przyznanie autorstwa jej mężowi Warrenowi.},
należy mieć na uwadze czytając to~co pisze on~o~I~Wojnie Światowej
na~stronach 867 i~873.

\vspace{\spaceFour}


\start \StrWd{299}{1} Czcionka w~tej linii jest za~duża.

\vspace{\spaceFour}


\start \StrWd{305}{4} Imię ojca Rasputina Efima, na~str.~313 jest
pisane ,,Jefim''.

\vspace{\spaceFour}


\start \StrWd{352}{1} Czcionka w~tej linii jest za~duża.

\vspace{\spaceFour}


\start \Str{355} W~pierwszym paragrafie jest mowa o~głosowaniu które
zakończyło się wynikiem siedem do~pięciu, później zaś, że~decyzja
o~pokoju z~Niemcami przeszła stosunkiem siedem do~czterech. Najpewniej
w~obu przypadkach mowa jest o~tym samym głosowaniu i~jeden z~podanych
wyników jest błędny.

\vspace{\spaceFour}


\start \Str{383} Jeśli niczego nie przeoczyłem, to w~tym miejscu
ostatni raz jest mowa o~Denikinie i~jego armii, gdy wycofują~się
na~Kubań i~Krym. Nie dowiadujemy~się tym samym jakie były ich dalsze
losy.

\vspace{\spaceFour}


\start \Str{397} Ponieważ Polska, zapewne tak samo, jak kraje
nadbałtyckie, nie istniała w~1914~r., jest nieprawdopodobne, by
w~memorandum Erzberga była mowa o~nich jako o~sąsiadujących
z~Niemcami. Należy~się domyślać, że~Erzberg chciał włączenia
wszystkich ziem które można było uznać za w~jakimś sensie polskie,
analogicznie dla~państw nadbałtyckich, do~Cesarskich Niemiec po
wygranej wojnie.

\vspace{\spaceFour}


\start \Str{456} Na~dole strony pozostawiono puste miejsce, które
powinien zajmować tekst z~następnej strony.

\vspace{\spaceFour}


\start \StrWd{867}{17} Linia jest źle zedytowana. Drugie zdanie w~tej
linii jest początkiem następnej pozycji w~bibliografii, powinna więc
być zgodnie z~tym sformatowana.

\vspace{\spaceFour}



\newpage
\CenterTB{Błędy}
\begin{center}
  \begin{tabular}{|c|c|c|c|c|}
    \hline
    & \multicolumn{2}{c|}{} & & \\
    Strona & \multicolumn{2}{c|}{Wiersz}& Jest & Powinno być \\ \cline{2-3}
    & Od góry & Od dołu &  &  \\ \hline
    7 & & 4 & wszystko$^{ * }$ & wszystko \\
    7 & & 3 & Rekonkwiście$^{ * }$ & Rekonkwiście \\
    23 & & 10 & \emph{Vhutch} & \emph{Church} \\
    24 & & 25 & Zbawiciela$^{ *^{ * } }$ & Zbawiciela$^{ ** }$ \\
    25 & & 9 & 1919 & 1819 \\
    32 & 11 & & dostosowania & do~stosowania \\
    50 & & 12 & za~panowania & rozpoczęta za~panowania \\
    55 & & 12 & piętnstu & piętnastu \\
    55 & & 7 & interesy & interesy Południa \\
    67 & 17 & & bezbożności'')$^{ 31 }$ & bezbożności''$^{ 31 }$) \\
    67 & 8 & & północy & południa \\
    68 & 21 & & siom & siłom \\
    85 & 1 & & Herald'' Tribune'' & Herald'' \\
    96 & & 2 & W.H. Warren & W.H. Carroll \\
    97 & & 7 & ,,tak uważamy'' & ,,Tak uważamy'' \\
    104 & & 17 & i~związku & i~w~związku \\
    104 & & 2 & W.H.~Warren & W.H.~Carroll \\
    105 & & 11 & Counter-Revolution & Counter-Revolution'' \\
    105 & & 5 & W.H.~Warren & W.H.~Carroll \\
    116 & & 15 & stał~się był & stał~się \\
    117 & & 23 & wyd.3,Boston & wyd.~3, Boston \\
    121 & 3 & & tonizowały & uspokajały \\
    126 & & 4 & Pampelunie. & Pampelunie). \\
    127 & 10 & & aż~przez & potem aż~przez \\
    135 & & 12 & doskonalej & doskonałej \\
    137 & 6 & & go & je \\
    139 & & 5 & \emph{1833} & \emph{1883} \\
    141 & & 8 & tom~VI, rozdział~XIV & rozdział~VIII, \\
    141 & & 7 & P\emph{olitical} & \emph{Political} \\
    141 & & 5 & rozdział zatytułowany & rozdział~II, \\
    150 & 5 & & dogmatach; & dogmatach, \\
    151 & & 8 & torturom... & torturom. \\
    151 & & 7 & miasta.. & miasta. \\
    152 & 22 & & i~i & i \\
    165 & & 7 & 2003) & 2003 \\
    170 & 16 & & potomek & bratanek \\
    170 & & 10 & potomka, ,,F\"{u}hrera & potomka ,,F\"{u}hrera \\
    \hline
  \end{tabular}

  \begin{tabular}{|c|c|c|c|c|}
    \hline
    & \multicolumn{2}{c|}{} & & \\
    Strona & \multicolumn{2}{c|}{Wiersz}& Jest & Powinno być \\ \cline{2-3}
    & Od góry & Od dołu &  &  \\ \hline
    171 & & 7 & skrajnym wręcz & wręcz skrajnym \\
    177 & 3 & & ,,byliście & ,,Byliście \\
    187 & 21 & & roku~Na & roku. Na \\
    190 & & 1 & Karl Marx & \emph{Karl Marx} \\
    194 & 12 & & etc.. & etc. \\
    194 & & 8 & spikerze & mówcy \\
    197 & & 7 & 1987) & 1987 \\
    198 & 16 & & wojny; & wojny \\
    199 & & 2 & 210 & 210. \\
    208 & 7 & & jest & jest natomiast \\
    208 & & 5 & wschodni, wschodni & zachodni, wschodni \\
    210 & & 6 & Pratt,, & Pratt, \\
    210 & & 5 & Carrol & Carroll \\
    223 & & 20 & dna & dnia \\
    228 & & 4 & 2005) & 2005 \\
    229 & & 14 & \emph{Westrn} & \emph{Western} \\
    233 & & 11 & piaty & piąty \\
    235 & 16 & & rzecz & Rzecz \\
    235 & & 15 & Uranu & Urana \\
    237 & & 1 & 1954) & 1954 \\
    239 & 12 & & wyrazili & nie~wyrazili \\
    239 & 15 & & z~zatem & a~zatem \\
    243 & 5 & & uczony; & uczony. \\
    243 & 5 & & roku1743 & roku 1743 \\
    248 & & 1 & 2008) & 2008 \\
    265 & & 13 & si & się \\
    267 & & 3 & (1944 ) & (1944) \\
    270 & & 1 & \emph{s.} & s. \\
    273 & & 3 & 1944 & 1994 \\
    294 & 7 & & % ,,
                światu''... & światu... \\
    299 & & 11 & Kołłnotaj & Kołłontaj \\
    299 & & 1 & 1988) & 1988 \\
    303 & & 4 & % ,,
                \emph{opończa}'' & \emph{opończa} \\
    308 & & 19 & niszczysz & Niszczysz \\
    309 & 22 & & warstw & wszystkich warstw \\
    317 & & 3 & \emph{Kerensky; the} & \emph{Kerensky: The} \\
    320 & & 3 & Habsburg & \emph{Habsburg} \\
    324 & & 8 & eserowcow & eserowców \\
    \hline
  \end{tabular}

  \begin{tabular}{|c|c|c|c|c|}
    \hline
    & \multicolumn{2}{c|}{} & & \\
    Strona & \multicolumn{2}{c|}{Wiersz}& Jest & Powinno być \\ \cline{2-3}
    & Od góry & Od dołu &  &  \\ \hline
    327 & 7 & & w~coraz & ludzie w~coraz \\
    330 & & 1 & \emph{Wtnesses} & \emph{Witnesses} \\
    332 & & & Piotrogrodu,. & Piotrogrodu. \\
    338 & 22 & & roboty!, & roboty! \\
    349 & & 1 & \emph{s.} & s. \\
    351 & 4 & & 1917--1921 & 1914--1922 \\
    365 & & 5 & miasta ; & miasta; \\
    373 & 19 & & rok & rok. \\
    379 & & 2 & 1989) & 1989 \\
    380 & 7 & & zlej & złej \\
    380 & & 6 & destruktywna & destruktywną \\
    381 & & 1 & 1951) & 1951 \\
    385 & 10 & & dopływem & odpływem \\
    387 & 4 & & 1915--1922 & 1914--1922 \\
    387 & & 2 & 1989) & 1989 \\
    392 & & 3 & \emph{s.} & s. \\
    393 & & 6 & \emph{s.} & s. \\
    393 & & 2 & \emph{s.} & s. \\
    394 & 7 & & z & z~dala \\
    394 & & 1 & \emph{XV} , & \emph{XV}, \\
    396 & & 19 & Leonowi XII & Leonowi XIII \\
    % & & & & \\
    % & & & & \\
    % & & & & \\
    % & & & & \\
    % & & & & \\
    459 & & 2 & \emph{s.} & s. \\
    % & & & & \\
    487 & & 14 & John a.~Ryan & John A.~Ryan \\
    % & & & & \\
    % & & & & \\
    536 & & 3 & 56n & 56 \\
    553 & & 3 & \emph{1939--1940} , & \emph{1939--1940}, \\
    575 & & 3 & 1997) & 1997 \\
    % & & & & \\
    % & & & & \\
    \hline
  \end{tabular}

  \begin{tabular}{|c|c|c|c|c|}
    \hline
    & \multicolumn{2}{c|}{} & & \\
    Strona & \multicolumn{2}{c|}{Wiersz}& Jest & Powinno być \\ \cline{2-3}
    & Od góry & Od dołu &  &  \\ \hline
    858 & 5 & & Pio Non (bł.~Pius~IX): & \emph{Pio Non (bł.~Pius~IX):} \\
    858 & 19 & & \emph{ofCatholic} & \emph{of Catholic} \\
    858 & 19 & & \emph{History,}(St.~Louis & \emph{History} (St.~Louis \\
    859 & 13 & & portugalskiej & portugalskiej. \\
    860 & 5 & & DuffDavid. & Duff David \\
    861 & 7 & & państwa.. & państwa. \\
    861 & 18 & & wyd.. & wyd. \\
    862 & 6 & & S. John Brown & S., \emph{John Brown} \\
    862 & & 2 & Jen. & Jen, \\
    864 & 5 & & York, & York \\
    864 & 21 & & York, & York \\
    866 & & 2 & FDR & \emph{FDR} \\
    867 & 15 & & York, & York \\
    867 & & 2 & York, & York \\
    868 & 3 & & 2004.. & 2004. \\
    868 & 15 & & York, & York \\
    868 & 23 & & York, & York \\
    869 & 24 & & \emph{Denikin} & \emph{Denikin.} \\
    869 & & 12 & wojskowości.. & wojskowości \\
    870 & 8 & & 1937) & 1937). \\
    871 & 13 & & 1939 1961 & 1939, 1961 \\
    871 & 17 & & jedneaj & jednej \\
    873 & 1 & & York, & York \\
    873 & 9 & & York, & York \\
    873 & 10 & & York, & York \\
    873 & & 5 & 1958 1966 & 1958, 1966 \\
    874 & & 6 & Najlepsze I~najbardziej & Najlepsze i~najbardziej \\
    \hline
  \end{tabular}
\end{center}
\noi
\StrWg{103}{5} \\
\Jest mianem krucjaty (\emph{la~cruzada} ) określali \\
\Pow określali mianem krucjaty (\emph{la~cruzada}) \\
\StrWd{170}{10} \\
\Jest ,,F\"{u}hrera z~Poczdamu'', ojca Fryderyka Wielkiego \\
\Pow ,,F\"{u}hrera z~Poczdamu'', Fryderyka Williama~I, ojca Fryderyka
Wielkiego \\
\StrWd{228}{5} \\
\Jest The~Victory~of Reason: How Christianity Led to Freedom,
Capitalism
and~Western Success \\
\Pow \emph{The~Victory~of Reason: How Christianity Led to Freedom,
  Capitalism and~Western Success} \\
\StrWg{234}{18} \\
\Jest zapoczątkowujących teorię indukcji elektromagnetycznej \\
\Pow które doprowadziły do~powstania teorii indukcji elektromagnetycznej \\
\StrWd{237}{1} \\
\Jest Ford: The~Times, the~Man, and~the~Company \\
\Pow \emph{Ford: The~Times, the~Man, and~the~Company} \\
\StrWd{246}{9} \\
\Jest Alexander Graham Bell and~the~Passion for~Invention \\
\Pow \emph{Alexander Graham Bell and~the~Passion for~Invention} \\
\StrWd{299}{2} \\
\Jest Three Who Made a~Revolution \\
\Pow \emph{Three Who Made a~Revolution} \\
\StrWd{383}{18} \\
\Jest Kołczak \\
\Pow Kołczak doszedł do wniosku \\


\vspace{\spaceTwo}





% ####################
\Work{
  Red. E. Guerriero, M. Impagliazzo \\
  ,,Najnowsza historia Kościoła. Katolicy i~kościoły chrześcijańskie
  w~czasie pontyfikatu Jana Pawła II (1978--2005)'',
  \cite{GuerrieroImpagliazzoNajnowszaHistoriaKosciola06} }


% \CenterTB{Uwagi.}

% \start \StrWd{9}{6} Zamieszczony tu komentarz odnośnie słowa

\CenterTB{Błędy}
\begin{center}
  \begin{tabular}{|c|c|c|c|c|}
    \hline
    & \multicolumn{2}{c|}{} & & \\
    Strona & \multicolumn{2}{c|}{Wiersz}& Jest & Powinno być \\ \cline{2-3}
    & Od góry & Od dołu &  &  \\ \hline
    6 & 10 & & religia miały & nauka miały \\
    6 & & 10 & do & od \\
    7 & & 11 & dużo & duże \\
    14 & & 3 & zgodne & zgadzające~się \\
    30 & & 4 & Afryki & Ameryki Południowej \\
    51 & & 17 & śś. & św. \\
    63 & 9 & & 1987 & 1986 \\
    \hline
  \end{tabular}
\end{center}

\vspace{\spaceTwo}



% ####################
\Work{
  Ks. Bogusław Kumor \\
  ,,Historia Kościoła. Tom~I: Starożytność chrześcijańska'',
  \cite{KumorHistoriaKosciolaTomI03} }


\CenterTB{Błędy}
\begin{center}
  \begin{tabular}{|c|c|c|c|c|}
    \hline
    & \multicolumn{2}{c|}{} & & \\
    Strona & \multicolumn{2}{c|}{Wiersz}& Jest & Powinno być \\ \cline{2-3}
    & Od góry & Od dołu &  &  \\ \hline
    14 & 11 & & rzeciwieństwie & przeciwieństwie \\
    14 & 15 & & jednej formy & jedną formę \\
    % & & & & \\
    % & & & & \\
    % & & & & \\
    % & & & & \\
    % & & & & \\
    % & & & & \\
    \hline
  \end{tabular}
\end{center}

\vspace{\spaceTwo}





% ########################################
% \newpage
% \Field{Historia świecka}

% \vspace{\spaceThree}





% ##############################
\newpage
\Field{Historia Polski}

\vspace{\spaceTwo} \vspace{\spaceThree}
% ##############################



% ####################
\Work{
  Norman Davies \\
  ,,Serce Europy'', \cite{DaviesSerceEuropy14} }


\CenterTB{Uwagi}

\start \Str{16} Davies pisze tu o~tym, że~rzut trójwymiarowej Ziemi na
dwuwymiarową mapę musi pociągać za sobą zniekształcenia, jednak
mnie~się wydaje, że~pomieszał dwie rzeczy. Powierzchnię Ziemi można
uważać za dwuwymiarową, tak jak mapę i~to raczej nie jest problemem
w~kartografii.

Jednak jak wiadomo na mapie albo kąty albo kształt lądów nie mogą być
wiernie oddane. Jest to jednak związane z~Theorema Egregium Gaussa.
Jako szczególny przypadek wynika z~niego, że~takie zniekształcenia
przy przekształcaniu dwuwymiarowej powierzchni w~inną dwuwymiarową
powierzchnię muszą się pojawić, jeśli mają one różne krzywizny Gaussa,
dla~płaszczyzny zaś jest ona równa 0, a~dla sfery
$\frac{ 1 }{ r^{ 2 } }$.

\vspace{\spaceFour}


\start \StrWg{17}{12} W~tym miejscu Davies stosuje często spotykany
anachronizm pisząc o~Watykanie jako metonimie Ojca Świętego
i~najwyższych władz Kościoła. Do~czasu zajęcia przez wojska Victora
Emanuela~II Państwa Kościelnego w~1870~roku, władza doczesna papieża
nie ograniczała~się tylko do~tej dzielnicy Rzymu, lecz przez wieki
dotyczyła ogromnych połaci Półwyspu Apenińskiego. Dlatego trafniej
w~kontekście 1765~r. byłoby pisać o~decyzji Rzymu.

\vspace{\spaceFour}


\start \Str{28} Wydaje mi~się, że~nazwanie wszystkich wymienionych
tu~przez Davies państw dyktaturami, jest błędne. Państwo Watykańskie
i~Tybet Dalajlamy nie są demokracjami, ale~to za~mało, aby uznać je
za~dyktatury, bo~dyktatura jest pojęciem czysto negatywnym
i~piętnującym.

\vspace{\spaceFour}


\start \StrWg{52}{10} Zostawienie w~tym wersie słowa ,,Batman''
pisanego z~dużej litery, jest chyba błędem tłumacza. Sugeruje to,
że~chodzi o~Człowieka\dywiz Nietoperza, jednanego z~najbardziej
znanych superbohaterów amerykańskich, bardzo prawdopodobne jednak,
że~chodziło o~angielskie słowo ,,batman'' oznaczające ordynansa.

% \CenterTB{Błędy}
% \begin{center}
%   \begin{tabular}{|c|c|c|c|c|}
%     \hline
%     & \multicolumn{2}{c|}{} & & \\
%     Strona & \multicolumn{2}{c|}{Wiersz}& Jest & Powinno być \\ \cline{2-3}
%     & Od góry & Od dołu &  &  \\ \hline
%     & & & & \\
%     & & & & \\ \hline
%   \end{tabular}
% \end{center}

\vspace{\spaceTwo}





% ####################
\Work{
  Red. A. Nowak \\
  ,,Historie Polski w~XIX wieku. Kominy, ludzie i~obłoki: \\
  modernizacja i~kultura. Tom~I'', \cite{HistoriaPolskiXIXTomI13} }


\CenterTB{Błędy}
\begin{center}
  \begin{tabular}{|c|c|c|c|c|}
    \hline
    & \multicolumn{2}{c|}{} & & \\
    Strona & \multicolumn{2}{c|}{Wiersz}& Jest & Powinno być \\ \cline{2-3}
    & Od góry & Od dołu &  &  \\ \hline
    16 & & 15 & równości równość & równości \\
    % & & & & \\
    % & & & & \\
    % & & & & \\
    % & & & & \\
    \hline
  \end{tabular}
\end{center}

\vspace{\spaceTwo}





% ####################
\Work{
  Andrzej Nowak \\
  ,,Dzieje Polski. Tom~I do~1202: Skąd nasz ród'',
  \cite{NowakDziejePolskivI14a} }


\CenterTB{Uwagi}

\start \tb{Strona tytułowa.} W informacjach o~autorze jest podane,
że~był redaktorem naczelnym ,,ARCANA'' w~latach 1994--2012, lecz
prawidłowy okresem są chyba lata 1995--2012.

\vspace{\spaceFour}


\start \Str{62} W~ostatnim paragrafie jest mowa o~czterech królów
Słowian, ale~wymienionych jest tylko trzech.

\vspace{\spaceFour}

% POPRAW
\start \StrWg{85}{16} ,,Nie było innej drogi do~Europy w~końcu X~w.
jak poprzez chrzest, nie~było dalej w~Europie innej drogi do humanizmu
jak poprzez chrześcijaństwo.'' To zdanie chyba najlepiej oddaje
problem z~postrzeganiem chrześcijaństwa nie tylko przez Nowaka,
ale~i~przez przytłaczającą większość, jeśli nie~całość, polskiej myśli
patriotycznej. Chrześcijaństwo, nawet nie rzymski\dywiz katolicyzm,
jest tylko środkiem do osiągnięcia doczesnych, świeckich, ziemskich
celów, takich jak dostanie~się do~,,Europy'', albo kultywowanie
humanizmu, nie~zaś wiarą pochodzącą od Boga i~jedyną drogą do życia
wiecznego. To~zaś zredukowanie Boga i~wiary, do~doczesnych korzyści,
to~straszliwe zło.

\vspace{\spaceFour}


\start \StrWd{92}{21--20} Nie potrafię zrozumieć, czy chodziło o~to,
że~Otton~II przejściowo zdobył Akwizgran, czy że~miejsce to było
przejściowo stolicą cesarską.

\vspace{\spaceFour}


\start \Str{96} W~swoim wykładzie z~cyklu
\href{https://www.youtube.com/watch?v=QovVLT2fitc}{,,Filary Polskości:
  Mieszko i~Bolesław''} Nowak znacznie wyraźniej niż w~tej książce,
pokazał cały cynizm polityczny Mieszka~I. O~możliwości takiego
spojrzenia na tego władcę Nowak mówi tam otwarcie jednocześnie,
zarówno na wykładzie jaki i~książce, próbuje przykryć ten cynizm,
nazywając działania Mieszka ,,majstersztykiem polityki polskiej''.

\vspace{\spaceFour}


\start \StrWg{144}{24} Jest tu napisane, że~Miecław był cześnikiem,
jednak jaka była nadworna funkcja cześnika jest wyjaśnione dopiero
na~stronie~150.

\vspace{\spaceFour}


\start \StrWd{163}{17--15} Zdanie ,,Rozpoczyna od~modlitwy
do~św.~Piotra, patrona Stolicy Apostolskiej w~Rzymie i~zarazem
Gertrudowego syna, jadącego właśnie do~Rzymu, do~papieża.'' można
zrozumieć w~ten sposób, że~Gertruda modli~się do swojego syna, co jest
nonsensowne. Prawdopodobnie miało być ,,zarazem patrona Gerturdowego
syna'', co czyni całe zdanie zrozumiałym i~sensownym.


\CenterTB{Błędy}
\begin{center}
  \begin{tabular}{|c|c|c|c|c|}
    \hline
    & \multicolumn{2}{c|}{} & & \\
    Strona & \multicolumn{2}{c|}{Wiersz}& Jest & Powinno być \\ \cline{2-3}
    & Od góry & Od dołu &  &  \\ \hline
    41 & 3 & & W X w. jeszcze & Jeszcze w X w. \\
    53 & & 1 & Księga Wyjścia & Księga Rodzaju \\
    61 & 10 & & do dziejów & dla dziejów \\
    100 & 14 & & Kto & Kto to \\
    169 & 17 & & mnie & do mnie \\
    202 & & 1 & posiąść & przesiąść \\
    236 & 14 & & ,,Gall''). & ,,Gall''. \\
    % & & & & \\
    % & & & & \\
    % & & & & \\
    \hline
  \end{tabular}
\end{center}
\noi
\StrWg{61}{8} \\
\Jest z~Rocznika kapituły krakowskiej dawnego \\
\Pow z~dawnego Rocznika kapituły krakowskiej \\

\vspace{\spaceTwo}





% ####################
\newpage

\Work{
  Wojciech Roszkowski \\
  ,,Najnowsza historia Polski: 1914--1939'',
  \cite{RoszkowskiNajnowszaHistoriaPolski39-45Wyd11} }


\CenterTB{Uwagi}

\start Karygodną, i~to~niezależnie od uznawanej metodologi pisania
-prac historycznych, cechą całego tego wydania ,,Najnowszej historii
-Polski'', jest nieumieszczenie w~każdej tomie listy używanych
skrótów. -Należy dodać, że~jeśli skrót został wprowadzony w~jednym
tomie, to~nie -jest już wyjaśniany w~następnych, co~dodatkowo
komplikuje sprawę.

\vspace{\spaceFour}


\start Ciekawym wydaje~się zauważanie, że~w~tej książce Roszkowski
zrealizował chyba idealnie, jedno z~założeń zaprojektowanego przez
piłsudczyków programu edukacji, przyjętego po reformie
jędrzejewiczowskiej (1932): sprowadzenia lat 1918--1920 wyłączenie
do~tematu walki o~granice. Więcej na ten temat \\
w~Andrzej Chojnowski ,,Kwestia patriotyzmu w~poszukiwaniach
programowych obozu piłsudczykowskiego'', str.~136
\cite{PatriotyzmPolakow06}.

\vspace{\spaceFour}


\start \Str{26} Podany tu opis przyczyn wybuchu I~Wojny Światowej,
zwłaszcza bardzo silne stwierdzenie, że~Austro\dywiz Węgry
wypowiedziały wojnę Serbii pod naciskiem Niemiec, warto skonfrontować
z~tym co pisze M. Gilbert w~swojej książce na temat tego przedziwnego
wydarzenia \cite{Gilbert03}.

\vspace{\spaceFour}


\start \StrWd{78}{4} Cudzysłów otwarty w~tym wierszu nigdy nie został
zamknięty, przez co~nie wiadomo, gdzie~się kończy cytat.

\CenterTB{Błędy}
\begin{center}
  \begin{tabular}{|c|c|c|c|c|}
    \hline
    & \multicolumn{2}{c|}{} & & \\
    Strona & \multicolumn{2}{c|}{Wiersz}& Jest & Powinno być \\ \cline{2-3}
    & Od góry & Od dołu &  &  \\ \hline
    20 & & 9  & sita & siła \\
    27 & & 16 & z agrozić & zagrozić \\
    29 & 21 & & Bąjończyków & Bajończyków \\
    30 & 21 & & uczestniczyło w~niej & wśród jej członków było \\
    31 & & 8 & Hans Beseler & Hans von Beseler \\
    36 & 3  & & POW & POW. \\
    37 & 12 & & LLOYDA & LOYDA \\
    50 & 8  & & R.Dmowski & R.~Dmowski \\
    50 & 8  & & J.Molenda & J.~Molenda \\
    50 & 8  & & \emph{Pibudnczcy} & \emph{Piłsudczycy} \\
    50 & 21 & & \emph{Pobki} & \emph{Polski} \\
    68 & 14 & & j~ednolitego & jednolitego \\
    73 & & 13 & 1920 R & 1920 R. \\
    & & 17 & W braku & Z braku \\
    % & & & & \\
    % & & & & \\
    % & & & & \\
    % & & & & \\
    % & & & & \\
    % & & & & \\
    % & & & & \\
    % & & & & \\
    % & & & & \\
    % & & & & \\
    & & & & \\
    % & & & & \\
    % & & & & \\
    426 & 2 & & \emph{114} & 114 \\
    426 & 9 & & \emph{50} & 50 \\
    426 & 14 & & \emph{211} & 211 \\
    426 & & & & \\
    426 & & & & \\
    % & & & & \\
    427 & 4 & & \emph{389} & 389 \\
    427 & 7 & & \emph{170, 391} & 170, 391 \\
    427 & 8 & & \emph{114} & 114 \\
    \hline
  \end{tabular}
\end{center}
\noi
\StrWd{47}{6} \\
\Jest nadchodzącą zimą \\
\Pow nadchodzącej zimy \\

\vspace{\spaceTwo}






% #########################
\vspace{\spaceTwo}

\Work{Polska po~1939~r.}
% \Subfield{}

\vspace{\spaceTwo} \vspace{\spaceThree}
% #########################



% ####################
\Work{
  Red. Piotr Franaszka \\
  ,,Granice kompromisu. Naukowcy wobec aparatu władzy ludowej'',
  \cite{RedFranaszkaGraniceKompromisu15} }


\CenterTB{Uwagi}

Błędy:\\
\begin{center}
  \begin{tabular}{|c|c|c|c|c|}
    \hline
    & \multicolumn{2}{c|}{} & & \\
    Strona & \multicolumn{2}{c|}{Wiersz}& Jest & Powinno być \\ \cline{2-3}
    & Od góry & Od dołu &  &  \\ \hline
    9 & & 2 & A.Dziuba & A.~Dziuba \\
    17 & 14 & & ZMP$^{ 33 }$\ldots & ZMP$^{ 33 }$. \\
    % & & & & \\
    % & & & & \\
    % & & & & \\
    \hline
  \end{tabular}
\end{center}

\vspace{\spaceTwo}



% ####################
\Work{
  Wojciech Roszkowski \\
  ,,Najnowsza historia Polski: 1939--1945'',
  \cite{RoszkowskiNajnowszaHistoriaPolski39-45Wyd11} }


\CenterTB{Uwagi}

Błędy:\\
\begin{center}
  \begin{tabular}{|c|c|c|c|c|}
    \hline
    & \multicolumn{2}{c|}{} & & \\
    Strona & \multicolumn{2}{c|}{Wiersz}& Jest & Powinno być \\ \cline{2-3}
    & Od góry & Od dołu &  &  \\ \hline
    10 & & 5 & Wisy & Wisły \\
    % & & & & \\
    % & & & & \\
    % & & & & \\
    \hline
  \end{tabular}
\end{center}

\vspace{\spaceTwo}





% ####################
\Work{
  Wojciech Roszkowski \\
  ,,Najnowsza historia Polski: 1980--1989'',
  \cite{RoszkowskiNajnowszaHistoriaPolski80-89Wyd11} }


\CenterTB{Uwagi}

\start \Str{8} Jest to jeden z~największych przykładów straszliwie
suchej, pozbawiającej wydarzenia z~przeszłości realności, a~także nie
pozwalającej~się zorientować w~symbolach o~ogromnej wadze,
historiografii pozytywistycznej, chyba szkoły niemieckiej, jaką
reprezentuje Roszkowski. Jeśli dobrze wywnioskowałem z~tego
wystąpienia
\href{https://www.youtube.com/watch?v=6B93_3CCMac}{Sławomira
  Cenckiewicza}, to słynny skok Wałęsy przez płot, wedle niego był to
w~istocie mur, miał miejsce w~opisywanym tu dniu 22~sierpnia 1980~r.
Wałęsa musiał przeskoczyć ten mur, właśnie dlatego, że~spóźnił~się na
główne otwarcie stoczni. Tylko pomarzyć jak~by to opisał Paul Johnson.

\vspace{\spaceFour}


\start \Str{16} Choć nie~czytałem poprzednich tomów, i~być może jest
tam informacja o~tym kim jest Karol Modzelewski, to w~tym tomie jest
wymieniony tylko dwa razy, i~z tego powodu powinna być podana jakaś
informacja o~nim, by~czytelnik wiedział kto jest autorem tak wiele
znaczącej nazwy jak ,,Solidarność''.

\vspace{\spaceFour}


\start \StrWd{23}{6} Cudzysłów otwarty w~tym wierszu nigdy nie został
zamknięty, przez co~nie wiadomo, gdzie~się kończy cytat.

\vspace{\spaceFour}


\start \StrWd{26}{6} Nie jest podane co było tematem omawianej
tu~narady sztabowej.


% Błędy:\\
% \begin{center}
%   \begin{tabular}{|c|c|c|c|c|}
%     \hline
%     & \multicolumn{2}{c|}{} & & \\
%     Strona & \multicolumn{2}{c|}{Wiersz}& Jest & Powinno być \\ \cline{2-3}
%     & Od góry & Od dołu &  &  \\ \hline
%     & & & & \\
%     & & & & \\ \hline
%   \end{tabular}
% \end{center}

\vspace{\spaceTwo}





% ####################
\Work{
  Wojciech Roszkowski \\
  ,,Najnowsza historia Polski: 1980--1989'',
  \cite{RoszkowskiNajnowszaHistoriaPolski80-89Wyd11} }


\CenterTB{Uwagi}

\start \Str{8} Jest to jeden z~najgorszych przykładów straszliwie
suchej, pozbawiającej przeszłości realności, a~także nie
pozwalającej~się zorientować w~symbolach o~ogromnej wadze,
historiografii pozytywistycznej, chyba szkoły niemieckiej, jaką
reprezentuje Roszkowski. Jeśli dobrze wywnioskowałem z~tego
wystąpienia
\href{https://www.youtube.com/watch?v=6B93_3CCMac}{Sławomira
  Cenckiewicza}, to słynny skok Wałęsy przez płot, wedle Cenckiewicza
niego był to w~istocie mur, miał miejsce w~opisywanym tu dniu
22~sierpnia 1980~r. Wałęsa musiał przeskoczyć ten mur, właśnie
dlatego, że~spóźnił~się na główne otwarcie stoczni, kiedy wszedł by do
niej chowając~się w~tłumie tysięcy pracowników. Tylko pomarzyć jak~by
to opisał Paul Johnson.

\vspace{\spaceFour}


\start \Str{16} Choć nie~czytałem poprzednich tomów, i~być może jest
tam informacja o~tym kim jest Karol Modzelewski, to w~tym tomie jest
wymieniony tylko dwa razy i~nie da~się zrozumieć kim on jest. Powinna
być tu podana jakaś krótka informacja, kim on był, tak aby osoba
czytająca tylko ten tom, wiedział kto jest autorem tak wiele znaczącej
nazwy jak ,,Solidarność''.

\vspace{\spaceFour}


\start \StrWd{23}{6} Cudzysłów otwarty w~tym wierszu nigdy nie został
zamknięty, przez co~nie wiadomo, gdzie~się kończy cytat.

\vspace{\spaceFour}


\start \StrWd{26}{6} Nie jest podane co było tematem omawianej
tu~narady sztabowej.


% Błędy:\\
% \begin{center}
%   \begin{tabular}{|c|c|c|c|c|}
%     \hline
%     & \multicolumn{2}{c|}{} & & \\
%     Strona & \multicolumn{2}{c|}{Wiersz}& Jest & Powinno być \\ \cline{2-3}
%     & Od góry & Od dołu &  &  \\ \hline
%     & & & & \\
%     & & & & \\ \hline
%   \end{tabular}
% \end{center}

\vspace{\spaceTwo}





% ####################
\Work{
  Wojciech Roszkowski \\
  ,,Najnowsza historia Polski: 1989--2011'',
  \cite{RoszkowskiNajnowszaHistoriaPolski89-11Wyd11} }


% \CenterTB{Uwagi}

\CenterTB{Błędy}
\begin{center}
  \begin{tabular}{|c|c|c|c|c|}
    \hline
    & \multicolumn{2}{c|}{} & & \\
    Strona & \multicolumn{2}{c|}{Wiersz}& Jest & Powinno być \\ \cline{2-3}
    & Od góry & Od dołu &  &  \\ \hline
    23 & & 2 & na~naturalnym & za~naturalnym \\
    38 & & 11 & IX & XI \\
    45 & & 2 & przez & przed \\
    46 & & 15 & 2,5\%\% & 2,5\% \\
    65 & & 16 & wygrali & wyciągnęli \\
    % & & & & \\
    % & & & & \\
    % & & & & \\
    % & & & & \\
    % & & & & \\
    % & & & & \\
    % & & & & \\
    \hline
  \end{tabular}
\end{center}

\vspace{\spaceTwo}





% ####################
\Work{
  Paweł Zyzak \\
  ,,Efekt domina. Czy Ameryka obaliła komunizm w~Polsce? \\
  Tom~I'', \cite{ZyzakEfektDominaTomI16} }


% \CenterTB{Uwagi}

\CenterTB{Błędy}
\begin{center}
  \begin{tabular}{|c|c|c|c|c|}
    \hline
    & \multicolumn{2}{c|}{} & & \\
    Strona & \multicolumn{2}{c|}{Wiersz}& Jest & Powinno być \\ \cline{2-3}
    & Od góry & Od dołu &  &  \\ \hline
    11 & 4 & & Labor & \emph{Labor} \\
    17 & 5 & & miała obejmować & obejmować \\
    % & & & & \\
    % & & & & \\
    % & & & & \\
    % & & & & \\
    % & & & & \\
    \hline
  \end{tabular}
\end{center}
\noi
\tb{Tyla okładka, wiersz 17.} \\
\Jest Sorosa ,Williama \\
\Pow Sorosa, Williama \\

\vspace{\spaceTwo}





% ########################################
\newpage
\Field{Świat po~1914~r.}

\vspace{\spaceTwo} \vspace{\spaceThree}
% ########################################



% ####################
\Work{
  Martin Gilbert \\
  ,,Pierwsza wojna światowa'', \cite{GilbertPierwszaWojnaSwiatowa03} }


\CenterTB{Uwagi}

\start \StrWg{70}{2} Nie wiem kto popełnił błąd, żołnierz, autor czy
tłumacz, ale to zdanie o~martwym doboszu jest bez sensu.

\vspace{\spaceFour}


\start \StrWd{70}{5} To zdanie jest na~pewno źle przetłumaczone,
ale~nie~wiem jakie je poprawić.


\CenterTB{Błędy}
\begin{center}
  \begin{tabular}{|c|c|c|c|c|}
    \hline
    & \multicolumn{2}{c|}{} & & \\
    Strona & \multicolumn{2}{c|}{Wiersz}& Jest & Powinno być \\ \cline{2-3}
    & Od góry & Od dołu &  &  \\ \hline
    21 & 1 & & 1996 & 1993 \\
    68 & & 2 & Wielka Brytania & Rosja \\
    69 & 2 & & kulturowo & kulturowo'' \\
    69 & 16 & & 1 sierpnia & 12 sierpnia \\
    70 & 8 & & Pułk Feuchtingera, kiedy & Kiedy pułk Feuchtingera \\
    70 & & 4 & Sir Edward Gray, kiedy & Kiedy sir Edward Gray \\
    122 & & 10 & dal & dał \\
    177 & 15 & & ,,Walcie, aż lufy pękną''.
           & <<Walcie, aż lufy pękną>>''. \\
           % & & & & \\
    \hline
  \end{tabular}
\end{center}

\vspace{\spaceTwo}





% ####################
\Work{
  Paul Johnson \\
  ,,Historia świata XX wieku, od~Rewolucji Październikowej \\
  do~<<Solidarności>>. Tom~I'',
  \cite{JohnsonHistoriaSwiataXXWiekuvI09} }


\CenterTB{Uwagi}

\start \Str{17--18} Terminy id, ego, superego nie zostały wprowadzone
przez Freuda, lecz przez jego tłumacza, bądź tłumaczy, na~język
angielski. Sam Freud używał zwykłych słów z~języka niemieckiego:
das~Es, Ich, Uberich. (Powinno to być omówione w książce Burzyńskiej
i~Markowskiego \cite{BM09}).

\vspace{\spaceFour}


\start \Str{56} Książka Keynesa \emph{Ekonomiczne konsekwencje
  pokoju}, nie mogła ukazać~się pod koniec 1917 roku.
Najprawdopodobniej chodzi tu o~koniec roku 1919. Powinno~się tu też
znaleźć obszerniejsze omówienie treści tej książki.

\vspace{\spaceFour}


\start \Str{63} Głosowanie nad traktatem o~którym tu mowa
nie~odbyło~się w~marcu 1919 roku, lecz w~marcu 1920 r.

\vspace{\spaceFour}


\start \Str{72} Wyrażoną tu opinię, że~Polska skorzystała z~obawy
Wielkiej Brytanii przed zalewem bolszewizmu, warto skonfrontować z~tym
co wielokrotnie mówił
\href{https://www.youtube.com/watch?v=yfQ7rpq_irA}{Andrzej Nowak} i~co
opisał w~,,Pierwszej zdradzie zachodu''.

\vspace{\spaceFour}


\start \Str{90} Wydaje~się, że~opisany tu~ciąg przemówień Lenina
i~relacja Krupskiej o~tym jak położył~się spać bez słowa, dotyczą
wydarzeń z~jednego dnia, tego którego Lenin wrócił on do Rosji. Jeśli
to prawda powinno to zostać lepiej zaznaczone w~tekście, w~chwili
obecnej, nie jest to w~pełni jasne.

\vspace{\spaceFour}


\start \Str{189} Opis udziału niemieckich wojskowych w~zawieszeniu
przez Niemcy broni w~I~Wojnie Światowej, powinien być bardziej
wyczerpujący, w~tym momencie jest zbyt zwięzły, aby był jasny.

\vspace{\spaceFour}


\start \Str{189} W~tym miejscu po~raz pierwszy zostaje użyte
określenie ,,Alianci'' na~członków Ententy. Jest to~chyba anachronizm,
którego nie powinno~się stosować jako, że~nazwa ,,Aliantów'' jest
powszechnie przyjęta dla~sojuszu z~II, a~nie z~I, Wojny Światowej.

\vspace{\spaceFour}


\start \Str{223} Drugie zdanie drugiego paragrafu na~tej stronie jest
źle skonstruowane, nie wiem jednak jak je poprawić. Mimo tego, tego
jego sens jest jasny.

\vspace{\spaceFour}


\start \Str{232} Ludendorff spada tu jak z~nieba, aby zostać naczelnym
wodzem w~rządzie Hitlera, powołanym podczas puczu monachijskiego.
Warto byłoby napisać skąd on~się w~ogóle w~tym miejscu wziął.

\vspace{\spaceFour}


\start \Str{237--238} Tekst byłby znacznie bardziej logiczny, gdy
zamiast zdania ,,Poincar\'{e} manifestował arystokratyczną pogardę
dla~wulgarności klasy średniej i~francuskiego braku równowagi
emocjonalnej'', było ,,Poincar\'{e} manifestował arystokratyczną
pogardę dla~wulgarności klasy średniej i~francuski brak równowagi
emocjonalnej''.

\vspace{\spaceFour}


\start \Str{241} Stwierdzenie o~kurczącej~się populacji Francji jest
mało udane, bowiem jak zaraz potem Johnson wskazuje, populacja ta
w~rzeczywistości rosła. Problemem jest to, że~rosła ona bardzo słabo
w~porównaniu z~innymi państwami i~tym samym malał stosunek liczby
mieszkańców Francji do mieszkańców innych krajów w~Europie.

\vspace{\spaceFour}


\start \Str{262} Na tej stronie pojawia~się po~raz pierwszy
sformułowanie ,,teoria spisku'', którą lepiej byłoby zastąpić
przyjętym w~języku polskim terminem ,,teoria spiskowa''. Użycie
takiego słownictwa wynika zapewne z~tego, że~jeśli dobrze rozumiem,
książkę przetłumaczono jeszcze w~latach 80 XX~w., kiedy nie było
jeszcze w~języku polskim ustalonej nazwy na~to~zjawisko.

\vspace{\spaceFour}


\start \Str{288}

\vspace{\spaceFour}


\start \Str{304} W~2016~r. byłem na wykładzie na~temat chrześcijaństwa
w~Japonii w~XVI~wieku. Choć sam prowadzący przyznawał, że~w~tej
historii pewne kluczowe punkty są do~dziś niezrozumiałe, to
jednocześnie w~świetle wszystkich rzeczy o~jakich mówił, stwierdzenie,
że~chrześcijaństwo zostało odrzucone w~skutek kłótni misjonarzy, jest
gigantycznym uproszczeniem, a~może nawet wypaczeniem, historii.

\vspace{\spaceFour}


\start \Str{303} Czytając ten fragment odniosłem wrażenie, że~cesarz
Meiji był stary człowiekiem, gdy sprawował swą władzę,
w~rzeczywistości jednak objął formalnie panowanie, gdy miał 15 lat,
zmarł zaś w~wieku 60 lat. Jego następca cesarz Yoshihito urodził~się,
gdy miał on 27 lat, więc można wykluczyć wpływ wieku Meiji na stan
zdrowia jego następcy, co ten fragment mógł sugerować.

\vspace{\spaceFour}


\start \Str{305}

\vspace{\spaceFour}


\start \Str{310} Rok 1944 jako data przystąpienia Japonii do II~Wojny
Światowej jest błędny, od 1937~r. prowadziła już drugą wojnę
chińsko\dywiz japońską, zaś w~grudniu 1941 roku dokonała ataku na
Stany Zjednoczone. Jest to zapewne kolejna w~tej książce literówka,
nie wiem jednak jak~ją poprawić.

\vspace{\spaceFour}


\start \Str{345} Brak numeru strony w~prawy górnym rogu.

\vspace{\spaceFour}


\start \Str{345} Opisane tu wydarzenia, okupacja Korei przez
Japończyków i~ich reakcja na~sytuację w~Chinach powinny być
przedstawione szerzej. W~obecnej chwili przez swoją zwięzłość jest to
dosyć niejasne i~chaotyczne.

\vspace{\spaceFour}


\start \Str{454} Nie rozumiem dlaczego Stalin chował za~plecy prawą
rękę, skoro jego lewa była uszkodzona. Czy to dlatego, że~nie był
w~stanie schować lewej ręki za~plecami?

\vspace{\spaceFour}


\start \Str{457} W~tekście brak odwołania do~przypisu 11.

\vspace{\spaceFour}


\start \Str{459}

\vspace{\spaceFour}


\start \Str{465}

\vspace{\spaceFour}


\start \Str{477--478}

\newpage
\CenterTB{Błędy}
\begin{center}
  \begin{tabular}{|c|c|c|c|c|}
    \hline
    & \multicolumn{2}{c|}{} & & \\
    Strona & \multicolumn{2}{c|}{Wiersz}& Jest & Powinno być \\ \cline{2-3}
    & Od góry & Od dołu & & \\ \hline
    15 & 17 & & Mendla & prac Mendla \\
    28 & 8 & & cywilizacyjne & cywilizowane \\
    36 & 10 & & zastąpić & zaspokoić \\
    51 & & & Jedyny & Jeden \\ % Dokończ.
    54 & 11 & & [dotyczących planu) & [dotyczących planu] \\
    & & & ] & \\
    55 & & 1 & M. Keynes & J. M. Keynes \\
    64 & 2 & & kształt ów & ów kształt \\
    65 & & 14 & późniejszy doradca & doradca \\
    68 & 5 & & szśćdziesiątych & sześćdziesiątych \\
    82 & 6 & & Ghandi: & Ghandi. \\
    89 & 5 & & odbyć & przebiegać \\
    111 & 12 & & Ogó1norosyjski & Ogólnorosyjski \\
    138 & & 10 & socjalrewolucjonistom & socjalrewolucjoniści \\
    142 & & 8 & poszczegó1nymi & poszczególnymi \\
    154 & 11 & & spadl & spadł \\
    155 & & 15 & przemysł kluczowy & kluczowe gałęzie przemysłu \\
    160 & 14 & & ,,wolność pracy'' & <<wolność pracy>>'' \\
    161 & & 5 & upijać'' & upijać''. \\
    185 & & 16 & można & nie~można \\
    190 & 17 & & więc równe & równe \\
    221 & 3 & & był & nie był \\
    243 & 8 & & nie & nie została \\
    248 & & 3 & 1853 & 1853 -- przy. red.] \\
    252 & & 4 & Enqu\'{e}te sur la~monarchie
           & \emph{Enqu\'{e}te sur la~monarchie} \\
    259 & 14 & & Algierczyków & Algierczykom \\
    281 & & 2 & red. & red.] \\
    292 & 15 & & Jesteśmy & ,,Jesteśmy %''
    \\
    314 & & 6 & Korupcja & korupcja \\
    336 & 6 & & wiec & więc \\
    351 & 5 & & podejrzanego & ,,podejrzanego %''
    \\
    382 & 11 & & wzrosty & wzrosły \\
    392 & 10 & & dały & dawały \\
    393 & & 2 & Steffens,{ }{ }\emph{Individualism}
           & Steffens, \emph{Individualism} \\
    408 & & 13 & problem6w & problemów \\
    423 & 9 & & ulega kwestii & podlega dyskusji \\
    426 & & 7 & zalegle & zaległe \\ \hline
  \end{tabular}
\end{center}

\begin{center}
  \begin{tabular}{|c|c|c|c|c|}
    \hline
    & \multicolumn{2}{c|}{} & & \\
    Strona & \multicolumn{2}{c|}{Wiersz}& Jest & Powinno być \\ \cline{2-3}
    & Od góry & Od dołu & & \\ \hline
    427 & 4 & & Ministerstwa. Zdrowia & Ministerstwa Zdrowia \\
    436 & 1 & & wielkim & ,,wielkim % ''
    \\
    436 & & 13 & to jedynie & jedynie \\
    444 & 5 & & zaspokoić & uspokoić \\
    445 & & 10 & nie żądające & żądające \\
    449 & 9 & & Białym. Domu & Biały Domu \\
    457 & 9 & & Problemy & ,,Problemy % ''
    \\
    461 & & 10 & miłosierny!$^{ 20 }$ & miłosierny!''$^{ 20 }$. \\
    % & & & & \\
    % & & & & \\
    % & & & & \\
    \hline
  \end{tabular}
\end{center}
\noi \\
\tb{Okładka} \\
\Jest ''Solidarności'' \\
\Pow ,,Solidarności'' \\
\Str{1} \\
\Jest \tb{Historia świata} \\
\Pow \tb{Historia świata XX wieku} \\
\Str{3} \\
\Jest \tb{Historia świata} \\
\Pow \tb{Historia świata XX wieku} \\

\vspace{\spaceTwo}





% ########################################
\newpage
\Field{Świat po~1945~r.}

\vspace{\spaceTwo} \vspace{\spaceThree}
% ########################################



% ####################
\Work{
  Tony Judt \\
  ,,Powojnie. Historia Europy od~roku 1945'', \cite{JudtPowojnie16} }


\CenterTB{Uwagi}

\start \Str{28} Stwierdzenie, że~aż do lat trzydziestych~XIX~w. babcie
hiszpańskie straszyły dzieci Napoleonem, jest trochę niezręczne. Wojny
napoleońskie skończyły~się dopiero w~1815 roku, więc chodzi tu
o~wydarzenia sprzed 25--40 lat, co nie wydaje~się obecnie zbyt długim
czasem, choć możliwe, że~w~XIX wieku taka długa pamięć była czymś
niezwykłym. Jeśli to właśnie Judy chciał przekazać, to można było
to~zdanie sformułować lepiej.

\vspace{\spaceFour}


\start \StrWd{102}{8} Nie potrafię zrozumieć co~w~tym kontekście miało
znaczyć zdanie ,,co~wyjątkowo miało~się okazać nieskuteczne''.

\vspace{\spaceFour}


\start \Str{111} Warto byłoby podać trochę więcej informacji o~zimie
roku 1947, aby~pozwolić czytelnikom poczuć jej siłę. Kilka zadań
podających dokładnie temperaturę panującą wtedy w~Europie i~czas przez
jaki~się utrzymywała, byłoby zupełnie wystarczające.

\vspace{\spaceFour}


\start \StrWd{125}{7} Należałoby podać, w~jakiej walucie~są wyrażone,
przedstawione tu wydatki.

\vspace{\spaceFour}


\start \Str{131} W~ostatnim akapicie na~tej stronie jest mowa
o~frontach ludowych i~narodowych w~taki sposób, że~nie można zrozumieć
o~co tak naprawdę chodzi.

\vspace{\spaceFour}


\start \StrWg{150}{20} Zdanie ,,zbliża~się czas wielkich zawirowań
--~a~tym samym konieczność określenia przez Związek Radziecki
wynikających z~tego korzyści'' nie jest zbyt dobrze skonstruowane
i~trochę niezrozumiałe.

\vspace{\spaceFour}


\start \StrWg{159}{20} Zdanie ,,pozwoli Niemcom gnić, dopóki owoce
niemieckiej urazy i~beznadziei nie wpadną mu do~koszyka'' nie jest
ani~zbyt jasne, ani~nie brzmi zbyt dobrze w~języku polskim.

\vspace{\spaceFour}


\start \StrWg{174}{18} Słowa Milovana Dżilasa warto byłoby opatrzyć
komentarzem. W~tym momencie ich brzmienie jest trochę dziwne, a~sens
niepewny.

\vspace{\spaceFour}


\start \StrWg{197}{8} W~tym wierszu jest mowa o~przeżyciu przez
Brytyjczyków Pierwszej~Wojny Światowej, ale bardzo możliwe, że~jest
to~błąd i~tak naprawdę chodzi o~Drugą.

\vspace{\spaceFour}


\start \Str{204} Przypis konsultanta wydania polskiego, który obecnie
znajduje~się na~końcu zdania w~wierszu 21 od~góry, powinienem
znajdować~się na końcu zdania w~wierszy drugim od~góry.

\vspace{\spaceFour}


\start \StrWg{258}{17} Z~kontekstu ciężko wywnioskować, kim było
,,dwóch lewicowych członków ruchu oporu''. Ten fragment powinien być
poprawiony.

\vspace{\spaceFour}


\start \StrWd{282}{3} Na~końcu książki nie ma odniesienia do przypisu
z~tej linii.

\vspace{\spaceFour}


\start \StrWd{417}{19--18} O~Luckym Luku, czyli na polski Mającym
Szczęście Łukaszu, można stwierdzić wiele, ale~nie to,
że~jest~,,nieszczęsny''. Również uznanie tego komiksu za belgijski,
jest dla mnie kontrowersyjne.

\vspace{\spaceFour}


\start \Str{474} Raymond Aron był zapewne całe życie antykomunistą,
ale~ponieważ w~jednym z~wydań \emph{Opium dla~intelektualistów}
napisał, że~książkę można traktować jako marksistowską krytykę pewnych
zjawisk, jego stosunek do~intelektualnego dziedzictwa marksizmu,
pozostaje sprawą otwartą.

\vspace{\spaceFour}


\start \Str{510} Choć ,,Dziady. Część III'' zostały napisane po
Powstaniu Listopadowym, to jednak jego akcja rozgrywa~się kilka lat
przed tym wydarzeniem i~nie dotyczy losów powstańców, lecz losów ludzi
uciskanych przez władzę cara. Co~zresztą w~1968 roku również brzmiało
bardzo współcześnie. Należy zaznaczyć, że~trzeciej części ,,Dziadów''
nie można sprowadzić tylko do tego wątku, choć jest on jednym
z~najważniejszych.

\vspace{\spaceFour}


\start \StrWg{513}{7} W~1970 roku protesty o~szczególnych
konsekwencjach miały miejsce w~Szczecinie, nie można więc ich
ograniczać tylko do Gdańska.

\vspace{\spaceFour}


\start \StrWg{526}{14} Wydaje mi~się, że rozumiem sens zdania
o~optymistycznym zapatrzeniu w~postindustrialne wyobcowanie
i~bezduszność, ale według mnie nie powinno~się pisać w~taki przewrotny
i~skomplikowany sposób, aby czytelnik nie zgubił~się.

\vspace{\spaceFour}


\start \StrWg{556}{8} Nie rozumiem co miały znaczyć słowa
,,dla~obojętnych skrajów ruchu robotniczego''.

\vspace{\spaceFour}


\start \Str{561} Nie wiem czy mogę~się w~pełni zgodzić ze
stwierdzeniem, że~Foucault był w~głębi duszy racjonalistą. Możliwe,
lecz zapewne była to specyficzna odmiana racjonalizmu.

\vspace{\spaceFour}


\start \StrWd{585}{18} Panków to dzielnica Berlina, gdzie
w~początkowym okresie istnienia NRD~znajdowały się rezydencje władz
tego kraju.

\vspace{\spaceFour}


\start \StrWg{602}{13} Na~końcu książki nie ma odniesienia do przypisu
z~tej linii.

\vspace{\spaceFour}


\start \StrWd{655}{16--14} Zdanie ,,przeczyć fundamentalnemu
powinowactwu demokratycznego państwa opiekuńczego (bez względu na~to,
jak bardzo niewystarczająco) z~kolektywistycznym planem komunizmu''
jest napisane w~niezrozumiały sposób.

\vspace{\spaceFour}


\start \StrWd{681}{4--3} Nie rozumiem co miało dokładnie znaczyć
zdanie ,,Epoka zastoju Leonida Breżniewa (Michaił Gorbaczow)
żywiła~się wieloma złudzeniami''.

\vspace{\spaceFour}

\start \StrWd{681}{2} W~Polsce przyjęła~się pisonia tego nazwiska
,,Potiomkin'' nie jak w~krajach anglojęzycznych ,,Potemkin''.

\vspace{\spaceFour}


\start \StrWd{693}{16} Na~końcu książki nie ma odniesienia do przypisu
z~tej linii.

\vspace{\spaceFour}


\start \Str{694} Bill Clinton miał 46 lat, gdy został prezydentem USA
w~1993 roku, był więc młodszy od Michaił Gorbaczow który miał 54, gdy
został w~1985 roku Sekretarzem Generalnym KPZR. Jednak nie można
twierdzić, że~Gorbaczow był młodszy od każdego prezydenta USA do
Clintona, bowiem najmłodszym prezydentem do roku 2017, jest Theodore
Roosevelt który miał tylko 42 lata, gdy~objął ten urząd w~1901 roku.

\vspace{\spaceFour}


\start \StrWd{703}{18} Słowo ,,niezrównany'' brzmi trochę dziwnie
w~tym kontekście. Może powinno być ,,nierówny''.

\vspace{\spaceFour}


\start \StrWd{718}{2} Zwrot ,,bezwarunkowe niezrozumienie'' jest
dziwny i~ciężki do zrozumienia. Po~angielsku brzmiał zapewne
,,unconditional misunderstanding'', warto się zastanowić nad jego
lepszym tłumaczeniem.

\vspace{\spaceFour}


\start \StrWd{734}{19} NRD pojawia~się w~tym wersie dość
niespodziewanie, może chodziło o~Czechosłowację?

\vspace{\spaceFour}


\start \StrWd{742}{10} Na~końcu książki nie ma odniesienia do przypisu
z~tej linii.

\vspace{\spaceFour}


\start \StrWd{757}{6--5} Zdanie ,,którego rządzący byli
komunistycznymi satrapami, przejęli kontrolę nad tym obszarem'', brzmi
jakoś niezręcznie. Może dałoby~się je sformułować lepiej?

\vspace{\spaceFour}


\start \StrWd{782}{15} Powinno tu być wyjaśnione czym~są prawa
ciągnięcia.

\vspace{\spaceFour}


\start \StrWd{805}{9} Na~końcu książki nie ma odniesienia do przypisu
z~tej linii.

\vspace{\spaceFour}


\start \StrWd{823}{12-10} Zdanie ,,W~kraju było teraz więcej osób
mówiących po~holendersku niż~francusku (w~stosunku trzy do~dwóch),
które produkowały na~głowę mieszkańca i~zarabiał więcej.'' źle brzmi
i~nie od razu zrozumiałe.

\vspace{\spaceFour}


\start \tb{Wkładka~1, str.~6, u~góry.} Dla ułatwienia czytelnikom
orientacji, warto byłoby napisać, o~jaki akt stworzenia tu chodzi.

\vspace{\spaceFour}


\start \tb{Wkładka~1, str.~6, u~dołu.} Cytowane są tu te same słowa
Clementa Attlee co na stronie~127, jednak te dwie wersje nie są
identyczne.


\CenterTB{Błędy}

\nopagebreak

\begin{center}
  \begin{tabular}{|c|c|c|c|c|}
    \hline
    & \multicolumn{2}{c|}{} & & \\
    Strona & \multicolumn{2}{c|}{Wiersz}& Jest & Powinno być \\ \cline{2-3}
    & Od góry & Od dołu &  &  \\ \hline
     22 & 16 & & w~nich żyli & żyli w~nich \\
     34 & 10 & & --supermani & --~supermani \\
     35 & & 17 & prądu & braku prądu \\
     35 & & 16 & dość & lecz dość \\
     58 & & 7  & postępującą & postępującej \\
     58 & & 6  & degeneracją & degeneracji \\
     99 & 10 & & to & za~to \\
    127 & 11 & & \emph{metody} & \emph{metody dozwolone} \\
    131 & 13 & & radziecką & bolszewicką \\
    185 & 19 & & być może & może \\
    199 & 11 & & przeregulowanej prawnie & prawnie przeregulowanej \\
    199 & 18 & & miastem & miasto \\
    203 & & 4 & Europy Środkowej & na~Europę Środkową \\
    264 & 13 & & Odstawiliśmy & ,,Odstawiliśmy % ''
    \\
    356 & 16 & & męża & ojca \\
    380 & 12 & & k~o~n~t~r~rewolucji & k~o~n~t~r~r~e~w~o~l~u~c~j~i \\
    393 & 23 & & zwyczajowo tradycyjnie & zwyczajowo i~tradycyjnie \\
    409 & & 12 & manipulować. & manipulować''. \\
    409 & & 11 & dziesięcioleciu''. & dziesięcioleciu. \\
    430 & & 3 & specjalnością.: & specjalnością: \\
    457 & 15 & & (czy & czy \\
    457 & & 2  & wino. & wino''. \\
    478 & & 9  & dziewięćdziesiątych & sześćdziesiątych \\
    490 & & 14 & je! & je!'' \\
    509 & & 18 & uczeni & uczelni \\
    526 & & 5  & polityka & że~polityka \\
    555 & & 8  & się wydaje & wydaje~się \\
    564 & 9 & & and & i \\
    565 & 8 & & przeglądzie & w~przeglądzie \\
    582 & & 11 & Nemiec & Niemiec \\
    \hline
  \end{tabular}
\end{center}

\begin{center}
  \begin{tabular}{|c|c|c|c|c|}
    \hline
    & \multicolumn{2}{c|}{} & & \\
    Strona & \multicolumn{2}{c|}{Wiersz}& Jest & Powinno być \\ \cline{2-3}
    & Od góry & Od dołu &  &  \\ \hline
    596 & & 4 & Andreas) & Andreas \\
    611 & 5  & & 1976 & 1975 \\
    676 & 1  & & zresztą nie była & nie była zresztą \\
    682 & 12 & & zawłaszczenie odśrodkowego & odśrodkowe zawłaszczenie \\
    691 & & 3 & a~nawet & nawet \\
    707 & 2  & & na nowo & nowe \\
    710 & 12 & & pierwszy hotel & hotel \\
    711 & 19 & & był & nie~był \\
    718 & 2  & & działania & pracy \\
    793 & 7  & & przez & wobec \\
    793 & & 8 & międzynarodową... & międzynarodową. \\
    793 & & 3 & nawet & on nawet \\
    798 & & 6 & walczyć & wlec~się \\
    815 & 15 & & \emph{Beatrice Webb (1925)} & Beatrice Webb (1925) \\
    838 & 3  & & w~związku z~tym & w~tym czasie \\
    937 & & 13 & Tabu & Różne tabu \\
    967 & & 11 & góry(Schuman & góry (Schuman \\
    976 & 6 & & praktycznie & w~praktyce \\
    977 & & 7 & wpław & wpłat \\
    \hline
  \end{tabular}
\end{center}
\noi
\StrWd{110}{6} \\
\Jest na~samą perspektywę pokoju \\
\Pow na~samą myśl o~perspektywie pokoju \\
\StrWg{417}{15} \\
\Jest którymi~się w~nich chwalono \\
\Pow w~których~się nimi chwalono \\
\StrWg{555}{4} \\
\Jest ,,nowy patriotyzm'' za~granicą \\
\Pow ,,nowy patriotyzm'' \\
\StrWd{617}{2} \\
\Jest w~hiszpańskich przedsiębiorstwach \\
\Pow hiszpańskich przedsiębiorstw \\
\StrWd{698}{4} \\
\Jest który rozpada~się w~wartości 37~miliardów rozpadów na~sekundę \\
\Pow w~którym dochodzi do~37~miliardów rozpadów na~sekundę \\

\vspace{\spaceTwo}





% ########################################
\newpage
\Field{Świat po~1989~r.}

\vspace{\spaceTwo} \vspace{\spaceThree}
% ########################################



% ####################
\Work{
  J. Kofman, W. Roszkowski \\
  ,,Transformacja i postkomunizm'', \cite{KR99} }


\CenterTB{Błędy}

\begin{center}
  \begin{tabular}{|c|c|c|c|c|}
    \hline
    & \multicolumn{2}{c|}{} & & \\
    Strona & \multicolumn{2}{c|}{Wiersz}& Jest & Powinno być \\ \cline{2-3}
    & Od góry & Od dołu &  &  \\ \hline
    10 & & 17 & jedynie & jedynej \\
    13 & & 4 & o ekspansji & do ekspansji \\
    16 & 9 & & marntrawstwem & marnotrawstwem \\
    % & & & & \\
    % & & & & \\
    \hline
  \end{tabular}
\end{center}

\vspace{\spaceTwo}





% ####################
\Work{
  Philipp Ther \\
  ,,Nowy ład na starym kontynencie. Historia neoliberalnej \\
  Europy'', \cite{TherNowyLad15} }


\CenterTB{Uwagi}

\start \Str{120} Jest tu mowa, że~Lech Wałęsa był pierwszym
prezydentem demokratycznej polski. Trzeba sprawdzić, czy ten tytuł nie
powinien przypaść gen.~Wojciechowi Jaruzelskiemu.

% \vspace{\spaceFour}


\CenterTB{Błędy} \nopagebreak
\begin{center}
  \begin{tabular}{|c|c|c|c|c|}
    \hline
    & \multicolumn{2}{c|}{} & & \\
    Strona & \multicolumn{2}{c|}{Wiersz}& Jest & Powinno być \\ \cline{2-3}
    & Od góry & Od dołu &  &  \\ \hline
    13 & & 5 & Federalnej, & Federalnej. \\
    14 & 16 & & ,,ruskich'' & o~,,ruskich'' \\
    16 & 6 & & o & jako \\
    30 & 12 & & wschodniego & zachodniego \\
    33 & 2 & & \emph{prospects}'' & \emph{prospects} \\
    41 & 13 & & po1917 & po~1917 \\
    43 & 3 & & po1918 & po~1918 \\
    49 & & 2 & \emph{1989},, % ''
           & \emph{1989}, \\
    50 & & 13 & doprowadziło & nie~doprowadziło \\
    51 & & 1 & 61-65,71 & 61--65, 71 \\
    57 & & 2 & 1985--1988 & \emph{1985--1988} \\
    60 & & 7 & NRD.nCi & NRD. Ci \\
    60 & & 1 & \emph{War},: & \emph{War}, \\
    % & & & & \\
    62 & & 9 & przypieczętowa & przypieczętowano \\
    71 & & 8 & Niemczami~i & Niemcami~niż \\
    80 & & 17 & z~Lewobrzeżną & Lewobrzeżną \\
    85 & & 11 & w & o \\
    88 & & 6 & \tb{\emph{Anders}} & Anders \\
    109 & 1 & & & Tilly'emu \\
    124 & 11 & & 5) & 5). \\
    146 & 1 & & ( ang. & (ang. \\
    % & & & & \\
    % & & & & \\
    % & & & & \\
    % & & & & \\
    \hline
  \end{tabular}
\end{center}
\noi
\StrWd{30}{22} \\
\Jest metropolii''$^{ 14 }$. Natomiast w~regionalnej \\
\Pow metropolii''$^{ 14 }$, w~regionalnej \\

\vspace{\spaceTwo}





% ##############################
\newpage
\Field{Różne dzieła historyczne}

\vspace{\spaceTwo} \vspace{\spaceThree}
% ##############################



% ####################
\Work{
  Christopher A. Ferrara \\
  ,,Liberty: The God That Failed'', \cite{Ferrara12} }


\CenterTB{Błędy}

\begin{center}
  \begin{tabular}{|c|c|c|c|c|}
    \hline
    & \multicolumn{2}{c|}{} & & \\
    Strona & \multicolumn{2}{c|}{Wiersz}& Jest & Powinno być \\ \cline{2-3}
    & Od góry & Od dołu &  &  \\ \hline
    23 & 3 & & in & in an other \\
    40 & & 1 & \emph{St. Saint} & \emph{Saint} \\
    207 & & 2 & Government & \emph{Government} \\
    227 & 11 & & doing-not & doing--not \\
    % & & & & \\
    % & & & & \\
    % & & & & \\
    \hline
  \end{tabular}
\end{center}

\vspace{\spaceTwo}





% ####################
\newpage

\Work{
  Andrew Gordon \\
  ,,Nowożytna historia Japonii'',
  \cite{GordonNowozytnaHistoriaJaponii10} }


\CenterTB{Uwagi}

\start \StrWg{46}{3} Jest tu podana ilość mieszkańców Tokio w~1720~r.,
ale~do 1868~r. to miasto nosiło nazwę~Edo.

\vspace{\spaceFour}


\start \Str{83} Nie rozumiem na czym polegała dewaluacja złotych monet
przeprowadzona przez \emph{bakufu}, ani czemu wywołało to zwiększenie
podaży pieniądza i~inflację.

\vspace{\spaceFour}


\start \StrWg{85}{10} Nie jest napisane kim jest ten potężny
reformator i~wróg cudzoziemców.

\vspace{\spaceFour}


\start \Str{88} Powinno zostać tu dokładniej opisany zamach na Iiego
oraz jawnie napisane, czy zginął on w~tym zamachu, bądź w~skutek
niego. Z~kontekstu wynika, że~Iiego zginął.

\vspace{\spaceFour}


\start \Str{91}{15} Sens tego zdania miał być chyba następujący. Gdyby
rząd \emph{bakufu} przetrwał, to w~skutek jego reform powstałby system
zbliżony do tego, który stworzyła restauracja Meiji.

\vspace{\spaceFour}


\start \StrWd{134}{13--11} Nie jest wyjaśnione która grupa
z~wymienionych grup odziedziczyła z~okresu Tokugawów zachowania
ksenofobiczne.

\vspace{\spaceFour}


\start \StrWd{424}{11} Zgodnie z~tym co było napisane na~stronie~417
kobieta ta raczej nie fałszowała banknotów, lecz~potwierdzenia
depozytu z~lokalnej agencji kredytowej.

\vspace{\spaceFour}


\start \StrWg{425}{5} W~tym miejscu jest mowa o~sytuacji gospodarczej
Japonii na przełomie lat osiemdziesiątych i~dziewięćdziesiątych XX
wieku, należy więc zwrócić uwagę, że~Unia Europejska istnieje w~sensie
formalnym od~1~listopada 1993~roku. Może~się więc zdarzyć, że~podana
tu nadwyżka handlowa odnosi~się do~czasu, gdy Unia Europejska jeszcze
formalnie nie istniał i~użycie jej nazwy jest formą skrótu myślowego.

\vspace{\spaceFour}


\start \Str{432} Podany tu średni czas czas trwania kadencji premierów
Japonii są błędne bądź problematyczne. Według nich w~latach 1955--1989
było 12 premierów, przyjmując więc, że~pierwszym z~tej dwunastki jest
Hatoyama Ichir\^{o}\footnote{Hatoyama został premierem jeszcze w~1954
  roku, ale~ponieważ urzędował cały 1955 rok, uznałem, że~należy go
  wliczyć do tej listy. Do~obliczeń włączyłem pełen czas jego
  kadencji, co wydaje~się rozsądne, bo~powinno to dać mniej powodów
  do~zamieszania.}, a~ostatnim Takeshita Noboru, średni czas ich
kadencji to 2.9 roku, podczas, gdy mediana to tylko 2.2 roku. Jeśli
zaś rozważymy kwartyl górny $3/4$ to wynosi\footnote{Kwartyl górny
  wybrałem w~ten sposób, że~powyżej niego jest 25\% populacji, jego
  samego zaś umieszczam pośród pozostałych 75\%.} on~3.4. Wszystkie te
liczby są niższe od~podanej tu~średniej~3.7 roku.

Zauważmy jednak, że~gdyby przyjąć tak jak w~książce pisze, że~między
1955 a~1989 rokiem upłynęły 44 lata, jawny błąd, to średnia czasu
urzędowania rzeczywiście wychodzi 3.7 roku. Ja~przyjąłem, że~skoro
$1989 - 1955 = 34$ to taką długość należy przypisać temu okresowi,
oznacza to bowiem tylko kilkumiesięczny błąd w~sumie długości
kadencji.

Co do czasu urzędowania premierów w~latach 1989--2000, to aby uzyskać
liczbę 10 premierów, należy liczyć od~Takeshita Noboru, który zaczął
kadencję jeszcze w~1987~r., do~Mori Yoshir\^{o}, który zakończył ją
w~2001~r. Przyjęcie więc, że~dziesięciu premierów, sprawowało urząd
dwanaście lat, jest po~prostu bardzo niedokładnym postawieniem sprawy,
by~nie powiedzieć niechlujnym.

\CenterTB{Błędy}
\begin{center}
  \begin{tabular}{|c|c|c|c|c|}
    \hline
    & \multicolumn{2}{c|}{} & & \\
    Strona & \multicolumn{2}{c|}{Wiersz}& Jest & Powinno być \\ \cline{2-3}
    & Od góry & Od dołu &  &  \\
    \hline
    7 & 17 & & 398 & 298 \\
    29 & 5 & & nstąpiły & nastąpiły \\
    \hline
  \end{tabular}

  \begin{tabular}{|c|c|c|c|c|}
    \hline
    & \multicolumn{2}{c|}{} & & \\
    Strona & \multicolumn{2}{c|}{Wiersz}& Jest & Powinno być \\ \cline{2-3}
    & Od góry & Od dołu &  &  \\ \hline
    7 & 17 & & 398 & 298 \\
    29 & 5 & & nstąpiły & nastąpiły \\
    50 & 11 & & przepychały & przepychało \\
    50 & 19 & & sześć & sześćdziesiąt \\
    67 & 3 & & Kaitokud\^{o} & nad Kaitokud\^{o} \\
    77 & 4 & & prądów & tych prądów \\
    95 & 6 & & zamach & zamachem \\
    116 & & 12 & 1887 & 1867 \\
    123 & 17 & & ich właśnie & ich \\
    123 & & 10 & to & o~tym to \\
    146 & 4 & & 23~6475 & 236~475 \\
    156 & 1 & & osiemnastowieczni & dziewiętnastowieczni \\
    160 & 12 & & temat & łamach \\
    166 & 10 & & legacji & delegacji \\
    168 & & 5 & 265 & 264 \\
    190 & 11 & & kich & ich \\
    199 & & 7 & ,,kastowości % ''
           & ,,kastowości'' \\
    220 & & 4 & dawały & dodawały \\
    323 & 22 & & ojczyzny & Japonii \\
    326 & & 21 & główną & jednak główną \\
    368 & & 7 & wstrzymywania & utrzymywania \\
    380 & 25 & & pracy kluby & pracy, kluby \\
    432 & 16 & & czterdzieści cztery & trzydzieści cztery \\
    465 & 8 & & 56 & 256 \\
    465 & & 18 & Tokyo1963 & Tokyo 1963 \\
    \hline
  \end{tabular}
\end{center}
\noi
\StrWd{240}{7} \\
\Jest gospodarczej i~militarnej przewagi \\
\Pow o~gospodarczą i~militarną przewagę \\

\vspace{\spaceTwo}





% ####################
\Work{
  Paul Johnson \\
  ,,Narodziny nowoczesności'', \cite{JohnsonNarodzinyNowoczesnoci95} }


\CenterTB{Błędy}
\begin{center}
  \begin{tabular}{|c|c|c|c|c|}
    \hline
    & \multicolumn{2}{c|}{} & & \\
    Strona & \multicolumn{2}{c|}{Wiersz}& Jest & Powinno być \\ \cline{2-3}
    & Od góry & Od dołu &  &  \\ \hline
    29 & 2 & & cali, członie & cali, o członie \\
    142 & 1 & & Barbaji & Barbajowi \\
    142 & & 14 & w nową operą & z nową operą \\
    345 & 15 & & XIX & XVIII \\
    345 & 18 & & od & na od \\
    409 & 8 & & sposób & nie sposób \\
    % & & & & \\
    % & & & & \\
    % & & & & \\
    \hline
  \end{tabular}
\end{center}

\vspace{\spaceTwo}





% ##############################
\newpage
\Field{Biografie}

\vspace{\spaceTwo} \vspace{\spaceThree}
% ##############################



% ####################
\Work{
  Sławomir Cenckiewicz \\
  ,,Anna Solidarność'', \cite{CenckiewiczAnnaSolidarnosc10} }


\CenterTB{Błędy}
\begin{center}
  \begin{tabular}{|c|c|c|c|c|}
    \hline
    & \multicolumn{2}{c|}{} & & \\
    Strona & \multicolumn{2}{c|}{Wiersz}& Jest & Powinno być \\ \cline{2-3}
    & Od góry & Od dołu &  &  \\ \hline
    21 & 10 & & zajęłam & zajęła \\
    % & & & & \\
    \hline
  \end{tabular}
\end{center}

\vspace{\spaceTwo}





% ####################
\Work{
  Masha Gessen \\
  ,,Putin. Człowiek bez twarzy'',
  \cite{GessenPutinCzlowiekBezTwarzy12} }


\CenterTB{Uwagi}

\start \StrWd{9}{6} Zamieszczony tu komentarz odnośnie słowa
,,lustracja'', które ma wedle niego pochodzić z~greki, jest zapewne
wynikiem niedbałości tłumacza. Polskie słowo ,,lustracja'', pochodzi
najpewniej od słowa ,,lustro'', które wydaje~się w~ogóle nie związane
z~greką. Prawdopodobnie ten fragment został przetłumaczony
mechanicznie, bez~refleksji, że~w~języku polskim, w~przeciwieństwie
do~oryginału, ten związek etymologiczny nie zachodzi.

\vspace{\spaceFour}


\start \Str{75--76} W~przedstawionej tu opowieści jest pewna
niekonsekwencja. Na~75 stronie pisze, że~Putin był w~tłumie
drezdeńczyków nacierających na budynek Stasi, czyli musiał
znajdować~się na zewnątrz. Jednak na~następnej stronie pisze,
że~wyszedł do owego tłumu na zewnątrz, więc musiał znajdować~się
w~środku budynku. Nigdzie nie jest napisane, jak i~dlaczego opuścił
tłum i~wszedł do siedziby Stasi.

\vspace{\spaceFour}


\start \Str{93} Sacharow urodził~się w~1921 r., Gorbaczow zaś w~1931,
w~1989 mięli więc odpowiednio 68 i~58 lat. Nazwanie Gorbaczowa młody
to pewne nadużycie, wynikające zapewne z~kontrastu między schorowany,
bliski śmierci Sacharowem, a~pełnym energii Michaiłem.

\vspace{\spaceFour}


\start \Str{101} \tb{Akapit trzeci.} Powinno tu być jawniej napisane,
że~wracamy do historii Putina.

\vspace{\spaceFour}


\start \tb{Tylna okładka, wiersz 5 (od dołu).} Masha Gessen
urodziła~się w~1967~r. więc w~latach 1981--1991 miła od~14 do~24 lat,
jest więc wysoce nieprawdopodobne, by~w~tym okresie pracowała
w~Stanach Zjednoczonych. Zapewne chodziło o~to, że~wówczas tam
mieszkała.


\CenterTB{Błędy}
\begin{center}
  \begin{tabular}{|c|c|c|c|c|}
    \hline
    & \multicolumn{2}{c|}{} & & \\
    Strona & \multicolumn{2}{c|}{Wiersz}& Jest & Powinno być \\ \cline{2-3}
    & Od góry & Od dołu &  &  \\ \hline
    32 & 2 & & zdawali się nie & nie zdawali się \\
    48 & 7 & & przed wyznaczeniem & po wyznaczeniu \\
    59 & & 11 & założyciela & twórcy \\
    63 & 4 & & karierze$^{ 34 }$ & karierze \\
    63 & 6 & & n i c h''. & n i c h''$^{ 34 }$. \\
    63 & 9 & & twarze$^{ 35 }$ & twarze \\
    63 & 10 & & znaczenie''. & znaczenie''$^{ 35 }$. \\
    63 & & 17 & ważnego$^{ 36 }$ & ważnego \\
    63 & & 13 & KGB''. & KGB''$^{ 36 }$. \\
    64 & 2 & & międzyludzkich<<$^{ 37 }$ & międzyludzkich<< \\
    64 & 5 & & międzyludzkich'' & międzyludzkich''$^{ 37 }$ \\
    65 & 4 & & przyjaciółko$^{ 40 }$ & przyjaciółko \\
    78 & 5 & & robić$^{ 68 }$ & \\
    78 & 7 & & błędy?'' & błędy?''$^{ 68 }$ \\
    % & & & & \\
    % & & & & \\
    \hline
  \end{tabular}
\end{center}

\vspace{\spaceTwo}





% ##############################
\newpage
\Field{Historia nauki}

\vspace{\spaceTwo} \vspace{\spaceThree}
% ##############################



% ####################
\Work{
  Nicolas Bourbaki \\
  ,,Elementy historii matematyki'', \cite{NBEHM} }


\CenterTB{Błędy}
\begin{center}
  \begin{tabular}{|c|c|c|c|c|}
    \hline
    & \multicolumn{2}{c|}{} & & \\
    Strona & \multicolumn{2}{c|}{Wiersz}& Jest & Powinno być \\ \cline{2-3}
    & Od góry & Od dołu &  &  \\ \hline
    8 & 11 & & metodzie '' & metodzie'' \\
    82 & & 9 & \emph{metafizycznego}''; & \emph{metafizycznego}''); \\
    % & & & & \\
    % & & & & \\
    % & & & & \\
    \hline
  \end{tabular}
\end{center}

\vspace{\spaceTwo}





% ####################
\Work{
  C. B. Boyer \\
  ,,Historia rachunku różniczkowego i~całkowego \\
  i~rozwój jego pojęć'',
  \cite{BoyerHistoriaRachunkuRozniczkowegoICalkowego64} }


\CenterTB{Uwagi}

\start W~całej książce angielskie zwarte i~treściwe słowo ,,calculus''
jest zastąpione długim polskim terminem ,,rachunek różniczkowy
i~całkowy'', co często prowadzi do bardzo niezgrabnych stylistycznie
zdań. Lepiej byłoby wprowadzi do książki, obok powyższego, termin
,,analiza matematyczna'', który można ładnie skrócić do ,,analizy''.

\vspace{\spaceThree}


% ##########
\tb{Konkretne strony}

\vspace{\spaceFour}


\start \StrWd{18}{2} Umieszczenie w~tym samym zdaniu stwierdzenia
o~ścisłym sformułowaniu analizy już u~jej początków oraz faktu,
że~matematycy byli niewrażliwi na pewne subtelności, jest dość
karkołomne. Nie wspominając już o~tym, że~te ,,subtelności'' były
często bardzo poważne.

\vspace{\spaceFour}


\start \StrWd{19}{18} Użyte tu określenie ,,mistycyzm imaginacyjnej
spekulacji'' jest wyraźnie niesprawiedliwe w~stosunku do metafizyki,
najważniejszego działu filozofii. Nie~zmienia tego fakt, że~Boyer mógł
mieć na myśli tylko transcendentalną metafizykę ze~szkoły Kanta.

\vspace{\spaceFour}


\start \Str{23} Stwierdzenie, że pewne podstawowe idea zostały
usunięte z~analizy matematycznej, szerzej zaś, z~matematyki, są~mocno
wątpliwe.

\vspace{\spaceFour}


\start \StrWd{26}{6} Nazwanie podanych wyżej pojęć ,,sztucznymi'',
ciężko jest mi nazwać czymś innym, niż nieczułością na piękno
matematyki.

\vspace{\spaceFour}


\start \Str{28} Ponieważ drugie wydanie tej książki ukazało~się w~1949
r., autor nie mógł wiedzieć, że~w~latach 60 XX w., głównie za sprawą
prac Abrahama Robinsona zostanie sformułowana analiza niestandardowa,
oparta na ścisły pojęciu nieskończenie małych liczb.


\CenterTB{Błędy}
\begin{center}
  \begin{tabular}{|c|c|c|c|c|}
    \hline
    & \multicolumn{2}{c|}{} & & \\
    Strona & \multicolumn{2}{c|}{Wiersz}& Jest & Powinno być \\ \cline{2-3}
    & Od góry & Od dołu &  &  \\ \hline
    29 & & 4 & [(376] & ([376] \\
    42 & 13 & & [402 & [402] \\
    % & & & & \\
    % & & & & \\
    \hline
  \end{tabular}
\end{center}

\vspace{\spaceTwo}





% ####################
\Work{
  R. Rhodes \\
  ,,Jak powstała bomba atomowa'', \cite{Rhodes00} }


\CenterTB{Błędy}
\begin{center}
  \begin{tabular}{|c|c|c|c|c|}
    \hline
    & \multicolumn{2}{c|}{} & & \\
    Strona & \multicolumn{2}{c|}{Wiersz}& Jest & Powinno być \\ \cline{2-3}
    & Od góry & Od dołu &  &  \\ \hline
    719 & & 6 & 1993 & 1933 \\
    % & & & & \\
    % & & & & \\
    \hline
  \end{tabular}
\end{center}

\vspace{\spaceTwo}





% ####################
\Work{
  A. K. Wróblewski \\
  ,,Historia fizyki'', \cite{Wroblewski06} }


\CenterTB{Błędy}
\begin{center}
  \begin{tabular}{|c|c|c|c|c|}
    \hline
    & \multicolumn{2}{c|}{Wiersz} & & \\ \cline{2-3}
    Strona & Od góry & Od dołu & Jest & Powinno być \\
    & (kolumna) & (kolumna) & & \\ \hline
    203 & 3 (2) & & Jacob 'sGravesande'a & Jacob's Gravesande'a \\
    % & & & & \\
    \hline
  \end{tabular}
\end{center}

\vspace{\spaceTwo}





% ##############################
\newpage
\Field{Eseje i~publicystyka}

\vspace{\spaceTwo} \vspace{\spaceThree}
% ##############################



% ####################
\Work{
  Andrzej Nowak \\
  ,,Strachy i lachy. Przemiany polskiej pamięci 1982-2012'',
  \cite{Nowak12} }


\CenterTB{Uwagi}

\start \Str{47} T.~S.~Eliot jest na tej stronie nazwany ,,wielkim
poetą katolickim'', acz z~tego co wiem do Kościoła nigdy nie
przyszedł, zamiast tego dołączył do jakiegoś wyznania
anglokatolickiego. Zaś użycie przymiotnika ,,wielki'' w~odniesieniu to
tego poety, którego twórczości nie da~się czytać, jest już na~pewno
błędem.

\vspace{\spaceTwo}





% ####################
\Work{
  Andrzej Nowak \\
  ,,Intelektualna historia III~RP. Rozmowy z~lat 1990--2012'',
  \cite{Nowak13} }


\CenterTB{Błędy}
\begin{center}
  \begin{tabular}{|c|c|c|c|c|}
    \hline
    & \multicolumn{2}{c|}{} & & \\
    Strona & \multicolumn{2}{c|}{Wiersz}& Jest & Powinno być \\ \cline{2-3}
    & Od góry & Od dołu &  &  \\ \hline
    195 & 6 & & pomocą & pomocy \\
    579 & & 3 & osób. & osób, \\
    580 & & 10 & od & do \\
    665 & 7 & & The National Interest'' & ,,The National Interest'' \\
    % & & & & \\
    % & & & & \\
    \hline
  \end{tabular}
\end{center}
\noi \\
\tb{Przednia okładka, wiersz 14.} \\
\Jest \emph{ImperologicalStudies.APolishPerspective}(2011);
\emph{Czaswalki} \\
\Pow \emph{Imperological Studies. A Polish Perspective} (2011);
\emph{Czas walki} \\
\tb{Przednia okładka, wiersz 10 (od dołu).} \\
\Jest \ldots w~Brnie \\
\Pow w~Brnie \\

\vspace{\spaceTwo}





% ####################
\Work{
  Andrzej Nowak \\
  ,,Historia i~polityka'', \cite{NowakHistoriaIPolityka16} }


\CenterTB{Uwagi}

\start \Str{} Nowak popełni tu pewien błąd pisząc o~grze komputerowej
,,Dzikie Pola'', jest to standardowa stołowa gra RPG i~nie ma nic
wspólnego z~komputerem.


\CenterTB{Błędy}
\begin{center}
  \begin{tabular}{|c|c|c|c|c|}
    \hline
    & \multicolumn{2}{c|}{} & & \\
    Strona & \multicolumn{2}{c|}{Wiersz}& Jest & Powinno być \\ \cline{2-3}
    & Od góry & Od dołu &  &  \\ \hline
    6 & 2 & & 448 & 484 \\
    18 & 18 & & ,, kult & ,,kult \\
    % & & & & \\
    % & & & & \\
    % & & & & \\
    % & & & & \\
    99 & & 2 & (1918--1920) & (1918--2008) \\
    133 & & 23 & 1981 & 1918 \\
    135 & & 5 & 361 (przytoczony & 361. Przytoczony \\
    % & & & & \\
    % & & & & \\
    \hline
  \end{tabular}
\end{center}

\vspace{\spaceTwo}





% ####################################################################
% ####################################################################
\bibliographystyle{alpha} \bibliography{Bibliography}{}



\end{document}

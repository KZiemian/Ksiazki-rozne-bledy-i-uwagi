% Autor: Kamil Ziemian

% --------------------------------------------------------------------
% Podstawowe ustawienia i pakiety
% --------------------------------------------------------------------
\RequirePackage[l2tabu, orthodox]{nag} % Wykrywa przestarzałe i niewłaściwe
% sposoby używania LaTeXa. Więcej jest w l2tabu English version.
\documentclass[a4paper,11pt]{article}
% {rozmiar papieru, rozmiar fontu}[klasa dokumentu]
\usepackage[MeX]{polski} % Polonizacja LaTeXa, bez niej będzie pracował
% w języku angielskim.
\usepackage[utf8]{inputenc} % Włączenie kodowania UTF-8, co daje dostęp
% do polskich znaków.
\usepackage{lmodern} % Wprowadza fonty Latin Modern.
\usepackage[T1]{fontenc} % Potrzebne do używania fontów Latin Modern.



% ----------------------------
% Podstawowe pakiety (niezwiązane z ustawieniami języka)
% ----------------------------
\usepackage{microtype} % Twierdzi, że poprawi rozmiar odstępów w tekście.
\usepackage{graphicx} % Wprowadza bardzo potrzebne komendy do wstawiania
% grafiki.
\usepackage{verbatim} % Poprawia otoczenie VERBATIME.
\usepackage{textcomp} % Dodaje takie symbole jak stopnie Celsiusa,
% wprowadzane bezpośrednio w tekście.
\usepackage{vmargin} % Pozwala na prostą kontrolę rozmiaru marginesów,
% za pomocą komend poniżej. Rozmiar odstępów jest mierzony w calach.
% ----------------------------
% MARGINS
% ----------------------------
\setmarginsrb
{ 0.7in} % left margin
{ 0.6in} % top margin
{ 0.7in} % right margin
{ 0.8in} % bottom margin
{  20pt} % head height
{0.25in} % head sep
{   9pt} % foot height
{ 0.3in} % foot sep



% ------------------------------
% Często używane pakiety
% ------------------------------
\usepackage{csquotes} % Pozwala w prosty sposób wstawiać cytaty do tekstu.
\usepackage{xcolor} % Pozwala używać kolorowych czcionek (zapewne dużo
% więcej, ale ja nie potrafię nic o tym powiedzieć).



% ------------------------------
% Często używane pakiety
% ------------------------------
\usepackage{csquotes} % Pozwala w prosty sposób wstawiać cytaty do tekstu.
\usepackage{xcolor} % Pozwala używać kolorowych czcionek (zapewne dużo
% więcej, ale ja nie potrafię nic o tym powiedzieć).



% ----------------------------
% Pakiety napisane przez użytkownika.
% Mają być w tym samym katalogu to ten plik .tex
% ----------------------------
\usepackage{latexshortcuts}



% --------------------------------------------------------------------
% Dodatkowe ustawienia dla języka polskiego
% --------------------------------------------------------------------
\renewcommand{\thesection}{\arabic{section}.}
% Kropki po numerach rozdziału (polski zwyczaj topograficzny)
\renewcommand{\thesubsection}{\thesection\arabic{subsection}}
% Brak kropki po numerach podrozdziału



% ----------------------------
% Ustawienia różnych parametrów tekstu
% ----------------------------
\renewcommand{\arraystretch}{1.2} % Ustawienie szerokości odstępów między
% wierszami w tabelach.






% Koniec komend
% ############################





% ----------------------------
% Pakiet "hyperref"
% Polecano by umieszczać go na końcu preambuły.
% ----------------------------
\usepackage{hyperref} % Pozwala tworzyć hiperlinki i zamienia odwołania
% do bibliografii na hiperlinki.





% ####################################################################
% Początek dokumentu
\begin{document}
% ####################################################################



% ########################################
\Main{Historia nauki, błędy i~uwagi} % Tytuł całego tekstu

\vspace{\spaceTwo} \vspace{\spaceThree}
% ########################################



% ##################
\Work{ % Autor i tytuł dzieła
  Nicolas Bourbaki \\
  ,,Elementy historii matematyki'', \cite{NBEHM} }


\CenterTB{Błędy}
\begin{center}
  \begin{tabular}{|c|c|c|c|c|}
    \hline
    & \multicolumn{2}{c|}{} & & \\
    Strona & \multicolumn{2}{c|}{Wiersz} & Jest
                              & Powinno być \\ \cline{2-3}
    & Od góry & Od dołu & & \\
    \hline
    8  & 11 & & metodzie '' & metodzie'' \\
    82 & &  9 & \emph{metafizycznego}''; & \emph{metafizycznego}''); \\
    % & & & & \\
    % & & & & \\
    % & & & & \\
    \hline
  \end{tabular}
\end{center}

\vspace{\spaceTwo}





% ##################
\Work{ % Autor i tytuł dzieła
  C. B. Boyer \\
  ,,Historia rachunku różniczkowego i~całkowego \\
  i~rozwój jego pojęć'',
  \cite{BoyerHistoriaRachunkuRozniczkowegoICalkowego64} }


\CenterTB{Uwagi}

\start W~całej książce angielskie zwarte i~treściwe słowo ,,calculus''
jest zastąpione długim polskim terminem ,,rachunek różniczkowy
i~całkowy'', co często prowadzi do bardzo niezgrabnych stylistycznie
zdań. Lepiej byłoby wprowadzi do książki, obok powyższego, termin
,,analiza matematyczna'', który można ładnie skrócić do ,,analizy''.

\vspace{\spaceThree}



% ##########
\tb{Konkretne strony}

\vspace{\spaceFour}


\start \StrWd{18}{2} Umieszczenie w~tym samym zdaniu stwierdzenia
o~ścisłym sformułowaniu analizy już u~jej początków oraz faktu,
że~matematycy byli niewrażliwi na pewne subtelności, jest dość
karkołomne. Nie wspominając już o~tym, że~te ,,subtelności'' były
często bardzo poważne.

\vspace{\spaceFour}


\start \StrWd{19}{18} Użyte tu określenie ,,mistycyzm imaginacyjnej
spekulacji'' jest wyraźnie niesprawiedliwe w~stosunku do metafizyki,
najważniejszego działu filozofii. Nie~zmienia tego fakt, że~Boyer mógł
mieć na myśli tylko transcendentalną metafizykę ze~szkoły Kanta.

\vspace{\spaceFour}


\start \Str{23} Stwierdzenie, że pewne podstawowe idea zostały
usunięte z~analizy matematycznej, szerzej zaś, z~matematyki, są~mocno
wątpliwe.

\vspace{\spaceFour}


\start \StrWd{26}{6} Nazwanie podanych wyżej pojęć ,,sztucznymi'',
ciężko jest mi nazwać czymś innym, niż nieczułością na piękno
matematyki.

\vspace{\spaceFour}


\start \Str{28} Ponieważ drugie wydanie tej książki ukazało~się w~1949
r., autor nie mógł wiedzieć, że~w~latach 60 XX w., głównie za sprawą
prac Abrahama Robinsona zostanie sformułowana analiza niestandardowa,
oparta na ścisły pojęciu nieskończenie małych liczb.


\CenterTB{Błędy}
\begin{center}
  \begin{tabular}{|c|c|c|c|c|}
    \hline
    & \multicolumn{2}{c|}{} & & \\
    Strona & \multicolumn{2}{c|}{Wiersz} & Jest
                              & Powinno być \\ \cline{2-3}
    & Od góry & Od dołu & & \\
    \hline
    29 & & 4 & [(376] & ([376] \\
    42 & 13 & & [402 & [402] \\
    % & & & & \\
    % & & & & \\
    \hline
  \end{tabular}
\end{center}

\vspace{\spaceTwo}





% ##################
\Work{ % Autor i tytuł dzieła
  R. Rhodes \\
  ,,Jak powstała bomba atomowa'', \cite{Rhodes00} }


\CenterTB{Błędy}
\begin{center}
  \begin{tabular}{|c|c|c|c|c|}
    \hline
    & \multicolumn{2}{c|}{} & & \\
    Strona & \multicolumn{2}{c|}{Wiersz} & Jest
                              & Powinno być \\ \cline{2-3}
    & Od góry & Od dołu & & \\
    \hline
    719 & &  6 & 1993 & 1933 \\
    % & & & & \\
    % & & & & \\
    \hline
  \end{tabular}
\end{center}

\vspace{\spaceTwo}





% ##################
\Work{ % Autor i tytuł dzieła
  A. K. Wróblewski \\
  ,,Historia fizyki'', \cite{Wroblewski06} }


\CenterTB{Błędy}
\begin{center}
  \begin{tabular}{|c|c|c|c|c|}
    \hline
    & \multicolumn{2}{c|}{Wiersz} & & \\ \cline{2-3}
    Strona & Od góry & Od dołu & Jest & Powinno być \\
    & (kolumna) & (kolumna) & & \\ \hline
    203 & 3 (2) & & Jacob 'sGravesande'a & Jacob's Gravesande'a \\
    % & & & & \\
    \hline
  \end{tabular}
\end{center}

\vspace{\spaceTwo}






% ####################################################################
% ####################################################################
% Bibliografia
\bibliographystyle{alpha} \bibliography{Bibliography}{}


% ############################

% Koniec dokumentu
\end{document}
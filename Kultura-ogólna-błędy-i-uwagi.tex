\RequirePackage[l2tabu, orthodox]{nag}
\documentclass[a4paper]{article}
\usepackage[polish]{babel}
\usepackage[utf8]{inputenc}
\usepackage[MeX]{polski}
\usepackage{microtype}
\let\lll\undefined
\usepackage[intlimits]{amsmath}
\usepackage{amsfonts, amssymb, amscd, amsthm}
\usepackage{verse}
\usepackage{xcolor}
\usepackage{calrsfs} % Lepsze kaligrafowane litery.
% \DeclareMathAlphabet{\pazocal}{OMS}{zplm}{m}{n}
\usepackage{vmargin}
% --------------------------------------------------------------------
% MARGINS
% --------------------------------------------------------------------
\setmarginsrb
{ 0.7in}  % left margin
{ 0.6in}  % top margin
{ 0.7in}  % right margin
{ 0.8in}  % bottom margin
{  20pt}  % head height
{0.25in}  % head sep
{   9pt}  % foot height
{ 0.3in}  % foot sep
\usepackage{hyperref}
% ####################################################################


\newcommand{\spaceOne}{2.5em}
\newcommand{\spaceTwo}{2em}
\newcommand{\spaceThree}{1em}
\newcommand{\spaceFour}{0.5em}

\newcommand{\ld}{\ldots}

\newcommand{\tb}{\textbf}
\newcommand{\noi}{\noindent}
\newcommand{\start}{\noi \tb{--} {}}
\newcommand{\Str}[1]{\tb{Str. #1.}}
\newcommand{\StrWg}[2]{\tb{Str. #1, wiersz #2.}}
\newcommand{\StrWd}[2]{\tb{Str. #1, wiersz #2 (od dołu).}}
\newcommand{\Dow}{\tb{Dowód}}
\newcommand{\Center}[1]{\begin{center} #1 \end{center}}
\newcommand{\CenterTB}[1]{\Center{\tb{#1}}}
\newcommand{\Jest}{\tb{Jest: }}
\newcommand{\Pow}{\tb{Powinno być: }}
\newcommand{\Pop}{{\color{red} Popraw.}}
\newcommand{\Prze}{{\color{red} Przemyśl.}}
\newcommand{\Dok}{{\color{red} Dokończ.}}
\newcommand{\Field}[1]{ \begin{center} {\Large \tb{#1} } \end{center} }
\newcommand{\Work}[1]{ \begin{center} {\large \tb{#1}} \end{center} }

\newcommand{\Main}[1]{ \begin{center} {\LARGE \tb{#1} } \end{center} }
\newcommand{\attribA}[1]{\nopagebreak{\vspace{2mm}\raggedleft
    \footnotesize #1\par\vspace{2em}}}
\newcommand{\attribB}[1]{\nopagebreak{\raggedleft\footnotesize #1
    \par \vspace{2em}}}
\verselinenumbersleft


\renewcommand{\arraystretch}{1.2}





% ####################################################################
\begin{document}
% ####################################################################



% ########################################
\Main{Kultura ogólna --~błędy i~uwagi.}

\vspace{\spaceThree}
% ########################################



% ####################
\Work{
  Artur Dmochowski \\
  ,,Kościół <<Wyborczej>>. Największa operacja resortowych dzieci.'',
  \cite{Dmo14}.}


\CenterTB{Błędy}
\begin{center}
  \begin{tabular}{|c|c|c|c|c|}
    \hline
    & \multicolumn{2}{c|}{} & & \\
    Strona & \multicolumn{2}{c|}{Wiersz}& Jest & Powinno być \\ \cline{2-3}
    & Od góry & Od dołu &  &  \\ \hline
    23 & 11 & & konstytucji(25 & konstytucji (25 \\
    29 & 12 & & nazwala & nazwała \\
    29 & 14 & & lat & lat>> \\ \hline
  \end{tabular}
\end{center}

\vspace{\spaceTwo}



\newpage





% ####################
\Work{
  E. Michael Jones \\
  ,,Zdeprawowani moderniści'', \cite{MJ14}.}


\CenterTB{Błędy}
\begin{center}
  \begin{tabular}{|c|c|c|c|c|}
    \hline
    & \multicolumn{2}{c|}{} & & \\
    Strona & \multicolumn{2}{c|}{Wiersz}& Jest & Powinno być \\ \cline{2-3}
    & Od góry & Od dołu &  &  \\ \hline
    & & & & \\
    16 & & 19 & Claya & Gaya \\
    24 & 4 & & człowieczeństwa & człowieczeństwa'' \\
    25 & 11 & & Samoa & ,,Samoa \\ %''
    25 & & 14 & wyłączności & Wyłączności \\
    27 & & 1 & beztroskich>> & beztroskich>>'' \\
    29 & 12 & & roku & roku. \\
    30 & & 9 & <<młodej studentki'' & <<młodej studentki>> \\
    30 & & 8 & bawełniane sukienki>> & <<bawełniane sukienki>> \\
    37 & 11 & & wówczas'' & wówczas \\
    37 & & 16 & niego & z~niego \\
    38 & & 2 & zbagatelizowałaś, & zbagatelizowałaś. \\
    49 & 12 & & za & z \\
    56 & 17 & & którym & których \\
    56 & 3 & & Wilberforce'a** & Wilberforce'a* \\
    67 & & 13 & \emph{Whay} & \emph{What} \\
    71 & 14 & & ,, Wartości %''
           & ,,Wartości \\ %''
    85 & & 13 & \emph{lalek} & \emph{lalek}'' \\
    91 & & 4 & a oni & ,,a oni \\ %''
    99 & 9 & & maja & mają \\
    101 & & 11 & naukowym & z naukowym \\
    103 & & 13 & ATA & ATS \\
    105 & 19 & & mniej silna & silniejsza \\
    107 & 19 & & Voris & Vorisem \\
    108 & 2 & & rzeczywistości & o rzeczywistości \\
    111 & 21 & & Indiana, & Indiana. \\
    111 & & 20 & w na & na \\
    119 & & 11 & oszukiwałam''$^{ 2 }$.W & oszukiwałam''$^{ 2 }$. W \\
    122 & 20 & & od & do \\
    161 & & 2 & wyraźn0e & wyraźne \\
    165 & & 3 & ktrego & którego \\
    171 & 12 & & zajmującysię & zajmujący~się \\
    186 & & 5 & umożliwiła & uniemożliwiła \\
    199 & & 15 & wiary & utraty wiary \\
    200 & & 1 & Mogło & Mogła \\
    201 & & 14 & roku & roku. \\
    205 & & 19 & Laetesa & Klaudiusza \\
    207 & 7 & & taka & taką \\
    211 & & 18 & ,,Szczyt & Szczyt \\ %''
    214 & 5 & & [ w~nagrodę] & [w~nagrodę] \\
    214 & 6 & & najwyraźniej .. & najwyraźniej\ld \\ \hline
  \end{tabular}
\end{center}

\begin{center}
  \begin{tabular}{|c|c|c|c|c|}
    \hline
    & \multicolumn{2}{c|}{} & & \\
    Strona & \multicolumn{2}{c|}{Wiersz}& Jest & Powinno być \\ \cline{2-3}
    & Od góry & Od dołu &  &  \\ \hline
    217 & 22 & & Fread & Freuda \\
    218 & & 2 & od~z & z \\
    226 & 10 & & Mannung & Manning \\
    228 & & 17 & pytanie:, & pytanie: \\
    230 & 12 & & 1963 roku & 1963 roku. \\
    230 & 13 & & 1425 & 1525 \\
    237 & & 4 & 560 & 1560 \\ \hline
  \end{tabular}
\end{center}
\noi \\
\StrWd{17}{10} \\
\Jest nieczystości pierworodnej córki jest ślepotą ducha. \\
\Pow pierworodną córką nieczystości jest ślepota ducha. \\
\StrWg{201}{8} \\
\Jest otorbił~się w~kokonie psychoanalizy samego siebie\ldots \\
\Pow otorbił samego siebie w~kokonie psychoanalizy\ldots \\
\StrWg{209}{20} \\
\Jest \emph{Gdyby potrafił być perwersyjny, byłby zdrowy, podobnie jak
  ojciec$^{ 202 }$}. \\
\Pow ,,Gdyby potrafił być perwersyjny, byłby zdrowy,
podobnie jak ojciec''$^{ 202 }$. \\

\vspace{\spaceTwo}





% ####################
\Work{
  Red. Piotr Kletowski,
  ,,Europejskie kino gatunków'', \cite{RedKletowski16}}


\CenterTB{Uwagi}

\start \StrWd{129}{1} W~filmie templariuszom nie wyłupiono oczu, lecz stracono i~powieszono na drzewach. Tam dzikie ptaki wyjadły ich oczy.

\CenterTB{Błędy}
\begin{center}
  \begin{tabular}{|c|c|c|c|c|}
    \hline
    & \multicolumn{2}{c|}{} & & \\
    Strona & \multicolumn{2}{c|}{Wiersz}& Jest & Powinno być \\ \cline{2-3}
    & Od góry & Od dołu &  &  \\ \hline
    87 & 11 & & wybraną & jedną wybraną \\
    % & & & & \\
    % & & & & \\
    % & & & & \\
    % & & & & \\
    \hline
  \end{tabular}
\end{center}
\noi
\StrWg{160}{12} \\
\Jest \emph{Za kilka dolarów więcej} Monco \\
\Pow filmu \emph{Dobry, zły i~brzydki} Blondie \\

\vspace{\spaceTwo}





% ####################
\Work{
  Zdzisław Krasnodębski \\
  ,,Zwycięzca po~przejściach: zebrane eseje i~szkice~V'',
  \cite{Krasnodebski12} }


\CenterTB{Uwagi}

\start \Str{317} Pominięto miejsce i~datę pierwszej publikacji
artykułu \emph{Nie udawaj Greka, Polsko!}

\vspace{\spaceTwo}





% ####################
\Work{
  Jan Sowa \\
  ,,Fantomowe ciało króla. Peryferyjne zmagania z~nowoczesną formą'',
  \cite{Sowa11} }


\CenterTB{Uwagi}

\start \StrWg{36}{15} Zdanie ,,jego aktywność ściśle wiąże~się
związana ze społeczną\ldots'', powinno brzmieć ,,jego aktywność ściśle
wiąże~się ze społeczną\ldots'' lub ,,jego aktywność jest ściśle
związana ze społeczną\ldots''. Choć na podstawie tekstu, nie można
wybrać wersji zamierzonej przez autora, nie~jest to jednak problemem,
bo obie przekazują tą samą treść.

\start \StrWg{63}{5--9} Zawarty tu tekst, jest źle skonstruowany
gramatycznie.

\CenterTB{Błędy}
\begin{center}
  \begin{tabular}{|c|c|c|c|c|}
    \hline
    & \multicolumn{2}{c|}{} & & \\
    Strona & \multicolumn{2}{c|}{Wiersz}& Jest & Powinno być \\ \cline{2-3}
    & Od góry & Od dołu &  &  \\ \hline
    92 & & 6 & 1140 & 1440 \\
    102 & 11 & & a bo & bo \\
    118 & 8 & & dawały & nie dawały \\
    300 & 17 & & mogły & mogło \\
    362 & 12 & & torrusa & torusa \\
    367 & 15 & & c\tb{oś} & \tb{coś} \\
    % & & & & \\
    % & & & & \\
    & & & & \\ \hline
  \end{tabular}
\end{center}

\vspace{\spaceTwo}





% \Work{
% M. Gessen \\
% ,,Putin. Człowiek bez twarzy.'', \cite{Ges12}.}


% \CenterTB{Uwagi.}

% \start \StrWd{9}{6} Zamieszczony tu komentarz odnośnie słowa
% ,,lustracja'', które ma wedle niego pochodzić z~greki, jest zapewne
% wynikiem niedbałości tłumacza. W~~oryginale wersji książki użyte tu
% słowo zapewne jest greckiego pochodzenia, ale polskie słowo
% ,,lustracja'', pochodzi najpewniej od słowa ,,lustro'', które
% wydaje~się w~ogóle nie związane z~greką. Prawdopodobnie ten fragment
% został przetłumaczony mechanicznie, bez~refleksji, że~w~języku
% polskim
% ten związek etymologiczny nie zachodzi. \\
% \start \Str{75--76} W~przedstawionej tu opowieści jest pewna
% niekonsekwencja. Na 75 stronie pisze, że~Putin był w~tłumie
% drezdeńczyków nacierających na budynek Stasi, czyli musiał
% znajdować~się na zewnątrz. Jednak na~następnej stronie pisze,
% że~wyszedł do owego tłumu na zewnątrz, więc musiał znajdować~się
% w~środku budynku. Nigdzie nie jest napisane, jak i~dlaczego opuścił
% tłum i~wszedł do siedziby Stasi. \\
% \start \Str{93} Należy sprawdzić w jakim wieku byli Gorbaczow i Sacharow w~1989 roku, bo nazwanie Gorbaczowa młodym, jakoś mi nie pasuje. \\
% \start \Str{101} \tb{Akapit trzeci.} Powinno tu być jawniej napisane, że~wracamy do historii Putina.  \\
% \start \tb{Tylna okładka, wiersz 5 (od dołu).} Masha Gessen urodziła~się w~1967~r. więc w~latach 1981--1991 miła od 14 do 24 lat, jest więc wysoce nieprawdopodobne, by~w~tym okresie pracowała w~Stanach Zjednoczonych. Zapewne chodziło o~to, że~wówczas tam mieszkała. \\


% \CenterTB{Błędy.}
% \begin{center}
%   \begin{tabular}{|c|c|c|c|c|}
%     \hline
%     & \multicolumn{2}{c|}{} & & \\
%     Strona & \multicolumn{2}{c|}{Wiersz}& Jest & Powinno być \\ \cline{2-3}
%     & Od góry & Od dołu &  &  \\ \hline
%     32 & 2 & & zdawali się nie & nie zdawali się \\
%     48 & 7 & & przed wyznaczeniem & po wyznaczeniu \\
%     59 & & 11 & założyciela & twórcy \\
%     63 & 4 & & karierze$^{ 34 }$ & karierze \\
%     63 & 6 & & n i c h''. & n i c h''$^{ 34 }$. \\
%     63 & 9 & & twarze$^{ 35 }$ & twarze \\
%     63 & 10 & & znaczenie''. & znaczenie''$^{ 35 }$. \\
%     63 & & 17 & ważnego$^{ 36 }$ & ważnego \\
%     63 & & 13 & KGB''. & KGB''$^{ 36 }$. \\
%     64 & 2 & & międzyludzkich<<$^{ 37 }$ & międzyludzkich<< \\
%     64 & 5 & międzyludzkich'' & międzyludzkich''$^{ 37 }$ \\
%     65 & 4 & & przyjaciółko$^{ 40 }$ & przyjaciółko \\
%     78 & 5 & & robić$^{ 68 }$ & \\
%     78 & 7 & & błędy?'' & błędy?''$^{ 68 }$ \\
%     & & & & \\
%     & & & & \\ \hline
%   \end{tabular}
% \end{center}




% \Work{
% Paul Johnson\\
% ,,Narodziny nowoczesności'', \cite{Joh95}.}


% \CenterTB{Błędy}

% \begin{center}
%   \begin{tabular}{|c|c|c|c|c|}
%     \hline
%     & \multicolumn{2}{c|}{} & & \\
%     Strona & \multicolumn{2}{c|}{Wiersz}& Jest & Powinno być \\ \cline{2-3}
%     & Od góry & Od dołu &  &  \\ \hline
%     & & & & \\
%     29 & 2 & & cali, członie & cali, o członie \\
%     142 & 1 & & Barbaji & Barbajowi \\
%     142 & & 14 & w nową operą & z nową operą \\
%     345 & 15 & & XIX & XVIII \\
%     345 & 18 & & od & na od \\
%     409 & 8 & & sposób & nie sposób \\ \hline
%   \end{tabular}
% \end{center}






% \Work{
% J. Kofman, W. Roszkowski \\
% ,,Transformacja i postkomunizm'', \cite{KR99}.}


% \CenterTB{Błędy}

% \begin{center}
%   \begin{tabular}{|c|c|c|c|c|}
%     \hline
%     & \multicolumn{2}{c|}{} & & \\
%     Strona & \multicolumn{2}{c|}{Wiersz}& Jest & Powinno być \\ \cline{2-3}
%     & Od góry & Od dołu &  &  \\ \hline
%     & & & & \\
%     10 & & 17 & jedynie & jedynej \\
%     13 & & 4 & o ekspansji & do ekspansji \\
%     16 & 9 & & marntrawstwem & marnotrawstwem \\
%     & & & & \\
%     & & & &  \\ \hline
%   \end{tabular}
% \end{center}


% \Work{
% Andrzej Nowak \\
% ,,Strachy i lachy. Przemiany polskiej pamięci 1982-2012.'',
% \cite{Now12}.}


% \Center{Uwagi:}

% \start \Str{47} T.~S.~Eliot jest na tej stronie nazwany ,,wielkim
% poetą katolickim'', acz z~tego co wiem do Kościoła nigdy nie
% przyszedł, zamiast tego dołączył do jakiegoś wyznania
% anglokatolickiego. Zaś użycie przymiotnika ,,wielki'' w~odniesieniu
% to tego poety, którego twórczości nie da~się czytać, jest już
% na~pewno błędem.


% % Błędy:\\
% % \begin{center}
% %   \begin{tabular}{|c|c|c|c|c|}
% %     \hline
% %     & \multicolumn{2}{c|}{} & & \\
% %     Strona & \multicolumn{2}{c|}{Wiersz}& Jest &
% %     Powinno
% %     być
% %     \\
% %     \cline{2-3}
% %     & Od góry & Od dołu & & \\ \hline
% %     & & & & \\
% %     & & & & \\ \hline
% %   \end{tabular}
% % \end{center}









% \Work{
% Red. A. Nowak \\
% ,,Historie Polski w~XIX wieku. Tom I: Kominy, ludzie i~obłoki:
% modernizacja i~kultura.'', \cite{HPXIX1}.}

% % Uwagi:
% % \begin{itemize}
% % \item[--] \Str{45} T.~S.~ Eliot jest na tej stronie nazwany
% %   ,,wielkim poetą katolickim'', acz z~tego co wiem do Kościoła
% %   nigdy
% %   nie przyszedł, zamiast tego dołączył do jakiegoś wyznania
% %   anglokatolickiego. Zaś przymiotnik ,,wielki'' w~odniesieniu to
% %   tego poety, którego twórczości nie da~się czytać, jest już
% %   na~pewno błędny.
% % \end{itemize}

% \CenterTB{Błędy}

% \begin{center}
%   \begin{tabular}{|c|c|c|c|c|}
%     \hline
%     & \multicolumn{2}{c|}{} & & \\
%     Strona & \multicolumn{2}{c|}{Wiersz}& Jest & Powinno być \\ \cline{2-3}
%     & Od góry & Od dołu &  &  \\ \hline
%     & & & & \\
%     16 & & 15 & równości równość & równości \\ \hline
%   \end{tabular}
% \end{center}



% \Work{
% R. Rhodes \\
% ,,Jak powstała bomba atomowa'', \cite{Rho00}.}


% Błędy:\\
% \begin{center}
%   \begin{tabular}{|c|c|c|c|c|}
%     \hline
%     & \multicolumn{2}{c|}{} & & \\
%     Strona & \multicolumn{2}{c|}{Wiersz}& Jest & Powinno być \\ \cline{2-3}
%     & Od góry & Od dołu &  &  \\ \hline
%     & & & & \\
%     719 & & 6 & 1993 & 1933 \\
%     & & & & \\ \hline
%   \end{tabular}
% \end{center}

% \Work{
% W. Roszkowski \\
% ,,Najnowsza historia Polski: 1914--1939'', \cite{Ros11a}.}


% \CenterTB{Uwagi:} \start Karygodną i to~niezależnie od uznawanej
% metodologi pisania prac historycznych, cechą całego tego wydania
% ,,Najnowszej historii Polski'', jest nieumieszczenie w~każdej tomie
% listy używanych skrótów,
% niezależnie od tego, czy zostały one wprowadzone w~tym, czy~w~którymś z~poprzednich. \\
% \start Ciekawym wydaje~się zauważanie, że~w~tej książce Roszkowski
% zrealizował chyba idealnie, jedno z~założeń zaprojektowanego przez
% piłsudczyków programu edukacji, przyjętego po reformie
% jędrzejewiczowskiej (1932): sprowadzenia lat 1918--1920 wyłączenie
% do~tematu walki o~granice. Więcej na ten temat \\ w~Andrzej Chojnowski ,,Kwestia patriotyzmu w~poszukiwaniach programowych obozu piłsudczykowskiego'', str. 136 \cite{PP06}. \\
% \start \Str{26} Podany tu opis przyczyn wybuchu I~Wojny Światowej,
% zwłaszcza bardzo silne stwierdzenie, że~Austro\dywiz Węgry
% wypowiedziały wojnę Serbii pod naciskiem Niemiec, warto
% skonfrontować z~tym co pisze M. Gilbert w~swojej książce na temat
% tego przedziwnego wydarzenia \cite{Gil03}. \\
% \start \StrWd{78}{4} Cudzysłów otwarty w~tym wierszu nigdy nie został zamknięty, przez co~nie wiadomo, gdzie~się kończy cytat. \\

% Błędy:\\
% \begin{center}
%   \begin{tabular}{|c|c|c|c|c|}
%     \hline
%     & \multicolumn{2}{c|}{} & & \\
%     Strona & \multicolumn{2}{c|}{Wiersz}& Jest & Powinno być \\ \cline{2-3}
%     & Od góry & Od dołu &  &  \\ \hline
%     & & & & \\
%     20 & & 9 & sita & siła \\
%     27 & & 16 & z agrozić & zagrozić \\
%     31 & & 8 & Hans Beseler & Hans von Beseler \\
%     36 & 3 & & POW & POW. \\
%     37 & 12 & & LLOYDA & LOYDA \\
%     47 & & 6 & przed nadchodzącą zimą & nadchodzącej zimy \\
%     50 & 21 & & \emph{Pobki} & \emph{Polski} \\
%     73 & & 13 & 1920 R & 1920 R. \\
%     & & 17 & W braku & Z braku \\
%     & & & & \\ \hline
%   \end{tabular}
% \end{center}


% \Work{
% A. K. Wróblewski \\
% ,,Historia fizyki'', \cite{Wro06}.}


% Błędy:\\
% \begin{center}
%   \begin{tabular}{|c|c|c|c|c|}
%     \hline
%     %     & \multicolumn{2}{c|}{} & & \\
%       & \multicolumn{2}{c|}{Wiersz} & & \\ \cline{2-3}
%       %       Strona & Od góry & Od dołu & Jest & Powinno być \\
%       & (kolumna) & (kolumna) & & \\ \hline
%       & & & & \\
%       %       203 & 3 (2) & & Jacob 'sGravesande'a & Jacob's
%       %       Gravesande'a \\
%       & & & & \\ \hline
%   \end{tabular}
% \end{center}





% ####################################################################
% ####################################################################
\bibliographystyle{alpha} \bibliography{Bibliography}{}



\end{document}

% Autor: Kamil Ziemian

% --------------------------------------------------------------------
% Podstawowe ustawienia i pakiety
% --------------------------------------------------------------------
\RequirePackage[l2tabu, orthodox]{nag} % Wykrywa przestarzałe i niewłaściwe
% sposoby używania LaTeXa. Więcej jest w l2tabu English version.
\documentclass[a4paper,11pt]{article}
% {rozmiar papieru, rozmiar fontu}[klasa dokumentu]
\usepackage[MeX]{polski} % Polonizacja LaTeXa, bez niej będzie pracował
% w języku angielskim.
\usepackage[utf8]{inputenc} % Włączenie kodowania UTF-8, co daje dostęp
% do polskich znaków.
\usepackage{lmodern} % Wprowadza fonty Latin Modern.
\usepackage[T1]{fontenc} % Potrzebne do używania fontów Latin Modern.



% ----------------------------
% Podstawowe pakiety (niezwiązane z ustawieniami języka)
% ----------------------------
\usepackage{microtype} % Twierdzi, że poprawi rozmiar odstępów w tekście.
\usepackage{graphicx} % Wprowadza bardzo potrzebne komendy do wstawiania
% grafiki.
\usepackage{verbatim} % Poprawia otoczenie VERBATIME.
\usepackage{textcomp} % Dodaje takie symbole jak stopnie Celsiusa,
% wprowadzane bezpośrednio w tekście.
\usepackage{vmargin} % Pozwala na prostą kontrolę rozmiaru marginesów,
% za pomocą komend poniżej. Rozmiar odstępów jest mierzony w calach.
% ----------------------------
% MARGINS
% ----------------------------
\setmarginsrb
{ 0.7in} % left margin
{ 0.6in} % top margin
{ 0.7in} % right margin
{ 0.8in} % bottom margin
{  20pt} % head height
{0.25in} % head sep
{   9pt} % foot height
{ 0.3in} % foot sep



% ------------------------------
% Często używane pakiety
% ------------------------------
\usepackage{csquotes} % Pozwala w prosty sposób wstawiać cytaty do tekstu.
\usepackage{xcolor} % Pozwala używać kolorowych czcionek (zapewne dużo
% więcej, ale ja nie potrafię nic o tym powiedzieć).





% --------------------------------------------------------------------
% Dodatkowe ustawienia dla języka polskiego
% --------------------------------------------------------------------
\renewcommand{\thesection}{\arabic{section}.}
% Kropki po numerach rozdziału (polski zwyczaj topograficzny)
\renewcommand{\thesubsection}{\thesection\arabic{subsection}}
% Brak kropki po numerach podrozdziału



% ----------------------------
% Ustawienia różnych parametrów tekstu
% ----------------------------
\renewcommand{\arraystretch}{1.2} % Ustawienie szerokości odstępów między
% wierszami w tabelach.





% --------------------------------------------------------------------
% Ogólne komendy zdefiniowane dla ułatwienia pracy z LaTeXem
% --------------------------------------------------------------------

% ##########
% Definicje odstępów w tekście
\newcommand{\spaceOne}{3em}
\newcommand{\spaceTwo}{2em}
\newcommand{\spaceThree}{1em}
\newcommand{\spaceFour}{0.5em}

% ##########
% Skróty do często używanych komend
\newcommand{\ld}{\ldots}
\newcommand{\tbs}{\textbackslash} % Backslash w tekście

% ##########
% Podstawowa komendy do edycja tekstu
\newcommand{\tb}{\textbf}
\newcommand{\noi}{\noindent}
\newcommand{\start}{\noi \tb{--} {}}

% Numeracja stron i rozdziałów
\newcommand{\Str}[1]{\tb{Str. #1.}}
\newcommand{\StrWg}[2]{\tb{Str. #1, wiersz #2.}}
\newcommand{\StrWd}[2]{\tb{Str. #1, wiersz #2 (od dołu).}}

% Do poprawiania błędów
\newcommand{\Dow}{\tb{Dowód}} % To niezbyt pasuje do opisu POPRAW
\newcommand{\Jest}{\tb{Jest: }}
\newcommand{\Pow}{\tb{Powinno być: }}

% Nagłówki
\newcommand{\Center}[1]{\begin{center} #1 \end{center}}
\newcommand{\CenterTB}[1]{\Center{\tb{#1}}}

\newcommand{\Main}[1]{ \begin{center} {\LARGE \tb{#1} } \end{center} }
\newcommand{\Field}[1]{ \begin{center} {\Large \tb{#1} } \end{center} }
\newcommand{\Work}[1]{ \begin{center} {\large \tb{#1}} \end{center} }

% Koniec komend
% ############################





% ----------------------------
% Pakiet "hyperref"
% Polecano by umieszczać go na końcu preambuły.
% ----------------------------
\usepackage{hyperref} % Pozwala tworzyć hiperlinki i zamienia odwołania
% do bibliografii na hiperlinki.






% ####################################################################
% Początek dokumentu
\begin{document}
% ####################################################################





% ########################################
\Main{DEUS --~błędy i~uwagi} % Tytuł całego tekstu

\vspace{\spaceTwo} \vspace{\spaceThree}
% ########################################



% ####################
\Work{
  Sanctae Jan Chryzostom \\
  ,,Homilie na~Księgę Rodzaju (seria pierwsza: Rdz 1--3)'',
  \cite{SJCh08} }


\CenterTB{Uwagi}

\start \Str{78} Na~końcu pierwszego akapitu jest jeden nieotwarty
cudzysłów, nie wiadomo więc, gdzie~się powinien zaczynać, a~gdzie
kończyć.

\vspace{\spaceFour}


\start \Str{106} Od końca pierwszego akapitu do~końca tej homilii,
tekst staję~się chaotyczny i~w~pewnym stopniu nielogiczny.
Należałoby~by sprawdzić, czy~został poprawnie przetłumaczony.


\CenterTB{Błędy}
\begin{center}
  \begin{tabular}{|c|c|c|c|c|}
    \hline
    & \multicolumn{2}{c|}{} & & \\
    Strona & \multicolumn{2}{c|}{Wiersz}& Jest & Powinno być \\ \cline{2-3}
    & Od góry & Od dołu &  &  \\ \hline
    73 & & 11 & w pokoju & pokoju \\
    % & & & & \\
    % & & & & \\
    \hline
  \end{tabular}
\end{center}

\vspace{\spaceTwo}





% ####################
\Work{
  Sanctae Augustyn z~Hippony \\
  ,,Państw Boże'', \cite{SAugCD} }


% \CenterTB{Uwagi}

% \start

\CenterTB{Błędy}
\begin{center}
  \begin{tabular}{|c|c|c|c|c|}
    \hline
    & \multicolumn{2}{c|}{} & & \\
    Strona & \multicolumn{2}{c|}{Wiersz}& Jest & Powinno być \\ \cline{2-3}
    & Od góry & Od dołu &  &  \\ \hline
    51 & 8 & & zostawia'' & zostawia \\
    92 & 16 & & Od & ,,Od \\ %''
    122 & 9 & & punicka. & punicka.'' \\
    % & & & & \\
    % & & & & \\
    % & & & & \\
    \hline
  \end{tabular}
\end{center}

\vspace{\spaceTwo}





% ####################
\Work{ % Autor i tytuł dzieła
  Sanctae Tomasz z Akwinu \\
  ,,O~królowaniu --~królowi Cypru'',
  \cite{SancteTomaszZAkwinuOKrolowaniu06} }


\CenterTB{Uwagi}

\start \StrWg{49}{1} Zdanie ,,Dalej: rzeczy zgodne z~naturą mają
w~sobie doskonałość, bowiem natura posługuje~się jednostkami --~i~tak
jest najlepiej'' powinno według mnie brzmieć raczej ,,Rzeczy zgodne
z~naturą mają w~sobie doskonałość: natura posługuje~się jednostkami
i~tak jest najlepiej''. Jednak bez~znajomości łaciny, nie da~się tego
problemu rozstrzyganą w~sposób merytoryczny. To~samo odnosi~się
do~wszystkich następnych uwaga o~sposobie tłumaczenie tekstu, chyba
że~powiedziano inaczej.

\vspace{\spaceFour}


\start \Str{63} Ta~strona sprawia dużo problemów, nie~wiadomo jednak
czy~jest to wina św.~Tomasza, czy tłumacza. Po~pierwsze w~punkcie~6.2
jest mowa o~,,dobre tyranii'', a~zgodnie z prowadzoną klasyfikacją,
nie może być czegoś takiego jak tyrania, która jest dobra. Po~drugie,
w~tym samym punkcie jest mowa, że~,,dobra tyrania'' nie unicestwia
pokoju jaki panuje w~społeczności. Jednak w~punkcie 4.9~na~stronie~55,
pisze św.~Tomasz, że~tyrani dla zachowania władzy zasiewają między
poddanymi niezgodę, a~tą która już istnieje podsycają.

\vspace{\spaceFour}


\start \StrWd{89}{3} Wydaj mi~się, że~zamiast ,,jeśli jest cnotliwe''
powinno być ,,jeśli jest własnością cnoty''

\vspace{\spaceFour}


\start \StrWd{91}{7} Popierając~się pobieżną i~niefachową analizą
tekstu łacińskiego, doszedłem do~wniosku, że~zamiast ,,ten sam porzuca
zwyczaj czynienia dobrze'' powinno być bardziej sensowne w~tym
kontekście zdanie ,,ten sam porzuca zwyczaj czynienia dobrze w~czasie
zamętu''.

\vspace{\spaceFour}


\start \StrWd{101}{13} Jestem słaby z~interpunkcji, mimo tego wydaje
mi~się, że~zamiast ,,a~kiedy już potrzeba więcej, oni dają królom
z~własnej woli'' powinno być ,,a~kiedy już potrzeba więcej oni dają
królom z~własnej woli''.

\vspace{\spaceFour}


\start \StrWg{103}{5} Zwykle w~tym wydaniu cytaty łacińskie~są
albo~przytaczane po łacinie, albo tłumaczone na~polski. W~tej linii
jednak, cytat jest przytoczony do~połowy po~łacinie, dalej jest
tłumaczenie całości na~polski.

\vspace{\spaceFour}


\start \StrWg{107}{6} Sytuacja taka sama jak na stronie~103, z~tym,
że~teraz po~łacinie przytoczona jest druga część cytatu.


\CenterTB{Błędy}
\begin{center}
  \begin{tabular}{|c|c|c|c|c|}
    \hline
    & \multicolumn{2}{c|}{} & & \\
    Strona & \multicolumn{2}{c|}{Wiersz}& Jest & Powinno być \\ \cline{2-3}
    & Od góry & Od dołu &  &  \\ \hline
    90  & & 10 & impune & \emph{impune} \\
    91  & &  1 & impune & \emph{impune} \\
    109 & & 11 & boskiego) & boskiego \\
    111 & &  1 & wszyst & wszyst\dywiz \\
    129 & & 11 & Policraticus & \emph{Policraticus} \\
    130 & & 19 & \emph{ypocrita} & \emph{hypocrita} \\
    % & & & & \\
    % & & & & \\
    % & & & & \\
    \hline
  \end{tabular}
\end{center}

\vspace{\spaceTwo}





% ####################
\Work{ % Autor i tytuł dzieła
  Sanctae Tomasz z Akwinu \\
  ,,Suma Teologiczna. Tom~I'', \cite{STomSTI} }


\CenterTB{Błędy}
\begin{center}
  \begin{tabular}{|c|c|c|c|c|}
    \hline
    & \multicolumn{2}{c|}{} & & \\
    Strona & \multicolumn{2}{c|}{Wiersz}& Jest & Powinno być \\ \cline{2-3}
    & Od góry & Od dołu &  &  \\ \hline
    32 & & 3 & bo\dywiz wiem & bowiem \\
    32 & & 2 & [Mądrość]dała & [Mądrość] dała \\
    34 & & 1 & za & ta \\
    35 & 13 & & ludzkiego''. & ludzkiego. \\
    36 & 3 & & jest & nie jest \\
    37 & & 14 & jest$^{ 2 }$ & jest''$^{ 2 }$ \\
    37 & 9 & & n & na \\
    38 & & 3 & biskupie & o biskupie \\
    38 & & 13 & Gdzie & ,,Gdzie \\ %''
    39 & & 1 & teologii ! & teologii! \\
    41 & 15 & & niewykształconych''$^{ 5 }$ & niewykształconych''$^{ 5 }$) \\
    45 & & 6 & pond & ponad \\
    48 & & 16 & druga & Druga \\
    49 & & 14 & poznania,{ }, & poznania, \\
    60 & & 16 & zak & tak \\
    % & & & & \\
    % & & & & \\
    % & & & & \\
    % & & & & \\
    % & & & & \\
    \hline
  \end{tabular}
\end{center}

\vspace{\spaceTwo}





% ####################
\Work{
  Sancte Franciszek Salezy \\
  ,,Filotea'', \cite{SFSF} }


\Center{Błędy}
\begin{center}
  \begin{tabular}{|c|c|c|c|c|}
    \hline
    & \multicolumn{2}{c|}{} & & \\
    Strona & \multicolumn{2}{c|}{Wiersz}& Jest & Powinno być \\ \cline{2-3}
    & Od góry & Od dołu &  &  \\ \hline
    10 & 8 & & prowadzićna & prowadzić na \\
    39 & 3 & & \emph{we mnie jest} & \emph{we mnie} \\
    39 & & 7 & rzeź\dywiz wić & rzeźwić \\
    40 & 10 & & szystkim & wszystkim \\
    45 & & 7 & twojemu & ich \\
    56 & & 9 & Ojczyznę.O, & Ojczyznę. O, \\
    59 & & 11 & światowych. & światowych.'' \\
    185 & & 12 & chciwy1 & chciwy$^{ 1 }$ \\
    263 & & 4 & ŚwiętyTomasz & Święty Tomasz \\
    % & & & & \\
    % & & & & \\
    % & & & & \\
    \hline
  \end{tabular}
\end{center}

\vspace{\spaceTwo}





% ####################
\begin{center}
  Bł. John Henry Newman\\
  ,,Apologia pro vita sua'',\\wydanie \romannumeral2.
\end{center}


% Uwagi:\\
Powinno być:
\begin{itemize}
\item[--] Str. 122. \ldots nad cały świat?
\item[--] Str. 139. \ldots słowem ,,Mądrości Bożej''''.
\item[--] Str. 176. \ldots wpływu praktycznego'', ,,Bogate
  wyposażenie\ldots %''
\item[--] Str. 193. ,,\emph{Via media} spała w bibliotekach\ldots %''
\item[--] Str. 199. \ldots Tomasza Cranmera (1489-1556).
  % \item[--] Str.
  % \item[--] Str.
  % \item[--] Str.
  % \item[--] Str.
  % \item[--] Str.
  % \item[--] Str.
  % \item[--] Str.
\end{itemize}

\vspace{\spaceTwo}





% ####################
\Work{
  Dante \\
  ,,Komedia'', \cite{DAK} }


\CenterTB{Błędy}
\begin{center}
  \begin{tabular}{|c|c|c|c|c|}
    \hline
    & \multicolumn{2}{c|}{} & & \\
    Strona & \multicolumn{2}{c|}{Wiersz}& Jest & Powinno być \\ \cline{2-3}
    & Od góry & Od dołu &  &  \\ \hline
    % 347 & 21 & & ,,Jestem & Jestem \\
    350 & & 9 & jed\dywiz nak & jednak \\
    % & & & & \\
    % & & & & \\
    % & & & & \\
    \hline
  \end{tabular}
\end{center}

\begin{itemize}
\item[--] Piekło, Pieśń IV, 131: \ldots którzy wiedzą,
  % \item[--] Str.
  % \item[--] Str.
  % \item[--] Str.
  % \item[--] Str.
  % \item[--] Czyściec, Pieśń 1, 40: ,,Dziw to zaiste\ldots
\item[--] Czyściec, Pieśń 1, 44: by was z tej nocy\ldots
  % \item[--] Str.
  % \item[--] Str.
\end{itemize}

\vspace{\spaceTwo}





% ####################
\Work{
  M. Davies \\
  ,,Sobór Papieża Jana. Rewolucja liturgiczna.'', \cite{Dav11} }

\CenterTB{Błędy}
\begin{center}
  \begin{tabular}{|c|c|c|c|c|}
    \hline
    & \multicolumn{2}{c|}{} & & \\
    Strona & \multicolumn{2}{c|}{Wiersz}& Jest & Powinno być \\ \cline{2-3}
    & Od góry & Od dołu &  &  \\ \hline
    11 & & 2 & \emph{musimocna} & \emph{musi mocno} \\
    41 & 10 & & ,,Patrząc & Patrząc %''
    \\
    46 & & 2 & Westminster & Westminster. \\
    % 81 & & 17 & ,,wyczuwalny & ,,Wyczuwalny \\
    97 & & 3 & wprowadzen\dywiz ia & wprowadzenia \\
    176 & & 9 & presji.,,Czy %''
           & presji. ,,Czy %''
    \\
    178 & 7 & & \emph{Novost}i & \emph{Novosti} \\
    183 & 5 & & protestanckich & prawosławnych \\
    188 & 14 & & wary & wiary \\
    203 & 14 & & podporządkować. & podporządkować.'' \\
    204 & & 15 & małżeństwa. & małżeństwa.'' \\
    209 & 3 & & zarazić. & zarazić.'' \\
    209 & 6 & & pracę.'' & pracę. \\
    243 & & 17 & pominiecie & pominięcie \\
    245 & 7 & & dokumentu. & dokumentu.'' \\
    250 & 17 & & nie byłaby & byłaby \\
    263 & 8 & & ,,<<Wzorcowa>>'' & ,,<<Wzorcowa>> %''
    \\
    265 & 13 & & tam, że & tam \\
    265 & 15 & & ewolucji, & ewolucji. \\
    310 & 5 & & obór & Sobór \\
    323 & 3 & & wyobrazili & wyobrażali \\
    323 & 11 & & nieuzasadnionym & nie uzasadnionym \\
    325 & & 8 & Nie & ,,Nie %''
    \\
    326 & 7 & & władzę. & władzę.'' \\
    327 & 3 & & 272 & 322 \\
    330 & 12 & & Podam -- nawet & Podam nawet \\
    330 & 13 & & to & jest \\
    359 & 3 & & 59\% & 41\% \\
    \hline
  \end{tabular}
\end{center}
\noindent\\
\StrWg{44}{16} \\
\Jest uprawnionych do głosowania \\
\Pow które można głosować \\
\StrWg{174}{4} \\
\Jest umożliwiaporozumieniezchrześcijaństwem.Strukturalnacałość\emph{DasKapital} \\
\Pow umożliwia porozumienie z chrześcijaństwem. Strukturalna całość
\emph{Das Kapital} \\

\vspace{\spaceTwo}





% ####################
\Work{ M. Davies \\
  ,,Sobór Watykański II a~wolność religijna'', \cite{Dav02} }


\CenterTB{Błędy}
\begin{center}
  \begin{tabular}{|c|c|c|c|c|}
    \hline
    & \multicolumn{2}{c|}{} & & \\
    Strona & \multicolumn{2}{c|}{Wiersz}& Jest & Powinno być \\ \cline{2-3}
    & Od góry & Od dołu &  &  \\ \hline
    84 & 2 & & społeczeństwie'' & ,,społeczeństwie'' \\
    109 & & 15 & nacjonalizmu & racjonalizmu \\
    118 & & 10 & był & został \\
    155 & 9 & & nie & się \\
    193 & 17 & & nie są & są \\
    258 & & 5 & \emph{rolą} & \emph{z rolą} \\
    262 & 17 & & \emph{upadku} & \emph{upadku.} \\
    269 & 12 & & \emph{Quanta cura} & \emph{Quas primas} \\
    269 & 17 & & \emph{Quanta cura} & \emph{Quas primas} \\
    333 & & 13 & jest & co jest \\
    % & & & & \\
    \hline
  \end{tabular}
\end{center}

\vspace{\spaceTwo}





% ####################
\Work{
  Tracey Rowland \\
  ,,Wiara Ratzingera, teologia Benedykta XVI'', \cite{Rowland10} }


\CenterTB{Uwagi}

\start \Str{50} Stwierdzenie, że~św. Jan Paweł II doszedł
do~teodramatyki poprzez tomizm i~fenomenologię, jest mocno wątpliwe.
O~ile pamiętam w~pierwszym artykule z~\cite{PM11}, jest podane, że
dopiero na studiach seminaryjnych poprzez standardowy podręcznik
metafizyki (?), Karol Wojtyła zetknął~się z~tomizmem. (Możliwe, że
w~tej pracy podane jest też, kiedy pierwszy raz spotkał~się
z~fenomenologią.) Natomiast co najmniej od lat szkolnych był
zafascynowany teatrem, str. 14--15 \cite{Now15a}. Zanim w~październiku
1942 r. podjął decyzję o zostaniu kapłanem był m.in.~od~kilku lat
zaangażowany w~działalność Teatru Rapsodycznego, z~którego twórcą
i~kierownikiem Mieczysławem Kotlarczykiem prowadził dyskusje, na
tematy wiążące, religię, filozofię, patriotyzm i~sztukę, patrz
np.~str.~19--34 w~\cite{Nowa15a}. Wydaje~się dużo bardziej
prawdopodobne, że~właśnie z~jego doświadczeń teatralnych, wyrosła jego
pasja do~teodramatyki.

\start \Str{52} Choć nie znam twórczości Ryszarda Legutko, wciąż
wydaje mi~się bardzo wątpliwym stwierdzeniem określenie go mianem
teologa. Osobną sprawą jest jego związek z~tradycją tomistyczną.

\CenterTB{Błędy}
\begin{center}
  \begin{tabular}{|c|c|c|c|c|}
    \hline
    & \multicolumn{2}{c|}{} & & \\
    Strona & \multicolumn{2}{c|}{Wiersz}& Jest & Powinno być \\ \cline{2-3}
    & Od góry & Od dołu &  &  \\ \hline
    21 & 16 & & 1844--1970 & 1844--1900 \\
    27 & & 9 & transcendentalne piękno & transcendentalnym pięknem \\
    46 & & 13 & 1947 & 1847 \\
    71 & & 11 & ,,przyjścia %''
           & >>przyjścia \\
    71 & & 7 & domu'' & domu<< \\
    78 & 12 & & ona z~tego & z~niej \\
    % & & & & \\
    % & & & & \\
    \hline
  \end{tabular}
\end{center}

\vspace{\spaceTwo}





% ####################
\Work{
  R. M. Wiltgen \\
  ,,Ren wpada do Tybru'', \cite{Wil10} }


% Uwagi:\\
% \begin{itemize}
%
% \item
%
% \item
%
% \end{itemize}

\CenterTB{Błędy}
\begin{center}
  \begin{tabular}{|c|c|c|c|c|}
    \hline
    & \multicolumn{2}{c|}{} & & \\
    Strona & \multicolumn{2}{c|}{Wiersz}& Jest & Powinno być \\ \cline{2-3}
    & Od góry & Od dołu &  &  \\ \hline
    18 & 5 & & dwunastu do czterech & dwunastu \\
    40 & & 16 & te & że \\
    42 & 3 & & Kantonu(Dahomej) & Kantonu (Dahomej) \\
    44 & & 6 & konferencji,tego & konferencji, tego \\
    % & & & & \\
    % & & & & \\
    % & & & & \\
    % & & & & \\
    \hline
  \end{tabular}
\end{center}





% ####################################################################
% ####################################################################
\bibliographystyle{alpha} \bibliography{Bibliography}{}



\end{document}

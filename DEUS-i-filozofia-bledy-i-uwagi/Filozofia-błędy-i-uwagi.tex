\RequirePackage[l2tabu, orthodox]{nag}
% Autor: Kamil Ziemian
\documentclass[a4paper,11pt]{article}
\usepackage[utf8]{inputenc}
\usepackage[polish]{babel}
\usepackage[MeX]{polski}
\usepackage{graphicx}
\usepackage{vmargin}
%----------------------------------------------------------------------
% MARGINS
%----------------------------------------------------------------------
\setmarginsrb
{ 0.7in}  % left margin
{ 0.6in}  % top margin
{ 0.7in}  % right margin
{ 0.8in}  % bottom margin
{  20pt}  % head height
{0.25in}  % head sep
{   9pt}  % foot height
{ 0.3in}  % foot sep
%\usepackage{amsfonts}% Czcionki matematyczne od American Mathematic Society.
%\usepackage{amsmath}% Dalsze wsparcie od AMS. Więc tego, co najlepsze w LaTeX, czyli trybu
%matematycznego.
%\usepackage{amscd}% Jeszcze wsparcie od AMS.
%\usepackage{latexsym}% Więcej symboli.
%\usepackage{textcomp}% Pakiet z dziwnymi symbolami.
%\usepackage{xy}% Pozwala rysować grafy.
%\usepackage{tensor}% Pozwala prosto używać notacji tensorowej. Albo nawet pięknej notacji
%tensorowej:).
\usepackage{hyperref}
% ####################################################################


% ####################
% Miary odstępów
\newcommand{\spaceOne}{2.5em}
\newcommand{\spaceTwo}{2em}
\newcommand{\spaceThree}{1em}
\newcommand{\spaceFour}{0.25em}


\newcommand{\ld}{\ldots}

% Edycja tekstu
\newcommand{\tb}{\textbf}
\newcommand{\noi}{\noindent}
\newcommand{\start}{\noi \tb{--} {}}
\newcommand{\Center}[1]{\begin{center} #1 \end{center}}
\newcommand{\CenterTB}[1]{\Center{\tb{#1}}}
\newcommand{\Str}[1]{\tb{Str. #1.}}
\newcommand{\StrWg}[2]{\tb{Str. #1, wiersz #2.}}
\newcommand{\StrWd}[2]{\tb{Str. #1, wiersz #2 (od dołu).}}
\newcommand{\Jest}{\tb{Jest: }}
\newcommand{\Pow}{\tb{Powinno być: }}
\newcommand{\Prze}{{\color{red} Przemyśl.}}
\newcommand{\Dok}{{\color{red} Dokończ.}}
\newcommand{\Main}[1]{ \begin{center} {\huge \tb{#1} } \end{center} }
\newcommand{\Field}[1]{ \begin{center} {\LARGE \tb{#1} } \end{center} }
\newcommand{\Work}[1]{ \begin{center} {\large \tb{#1}} \end{center} }


\renewcommand{\arraystretch}{1.2}



% #####################################################################
\begin{document}
% #####################################################################



% ########################################
\Main{Filozofia --~błędy i~uwagi}

\vspace{\spaceTwo} \vspace{\spaceThree}
% ########################################



% ##############################
\Field{Historia filozofii}

\vspace{\spaceFour}
% ##############################



% ####################
\Work{
  Frederick Copleston S. J. \\
  ,,Historia filozofii. Tom~I: Grecja i~Rzym'',
  \cite{CoplestonHistoriaFilozofiiTomI04} }


\CenterTB{Uwagi}

\noi \tb{Ogólne}

\vspace{\spaceFour}

\start Omówienie każde filozofa powinno zawierać listę dzieł
polecanych do przeczytania, aby~czytelnik wiedział, gdzie najlepiej
zetknąć~się z~tym wszystkim co~zawiera~się w~sposobie filozofowania
danego myśliciela. Z~rzeczami takimi jak konkretne sposoby dowodzenia,
używane argumenty, wplatane anegdoty, żarty, etc.

\vspace{\spaceThree}


\noi \tb{Konkretne strony}

\vspace{\spaceFour}

\start \Str{32} W~polskiej tłumaczeniu Diogenesa Laertios, opisuje
styl Anaksymenesa jako prosty i~niewyszukany, zaś w~wersji angielskiej
jako ,,pure unmixed'', co należałoby by przetłumaczyć raczej jako
,,czysty i~pozbawiony obcych naleciałości'', jednak to tłumaczenie
również jest niesatysfakcjonujące.

\vspace{\spaceFour}


\start \Str{33} Według tego co tu napisano, Anaksymanes uważał,
że~istnieje jedne pierścień okalający, zawierający i~ogień i~zimno,
wewnątrz niego zaś~jest powietrze. Nie potrafię sobie wyobrazić, jak
taki pierścień miałby wyglądać.

\vspace{\spaceFour}


\start \Str{33} Należałoby pójść śladem wydania angielskiego i~ostatni
akapit tej strony, który rozpoczyna podsumowanie filozofii jońskiej,
oddzielić graficznie od omówienia myśli Anaksymenesa.





\CenterTB{Błędy}
\begin{center}
  \begin{tabular}{|c|c|c|c|c|}
    \hline
    & \multicolumn{2}{c|}{} & & \\
    Strona & \multicolumn{2}{c|}{Wiersz}& Jest & Powinno być \\ \cline{2-3}
    & Od góry & Od dołu &  &  \\ \hline
    16 & & 3 & poglądów & spojrzenia na świat \\
    32 & 16 & & \ld Mówi & \ld mówi \\
    37 & & 18 & zgodę & niezgodę \\
    38 & 5 & & zwykłą & tylko \\
    % & & & & \\
    % & & & & \\
    61 & & 12 & do~Schelling & to~Schelling \\
    82 & 8 & & do~kraju & w~kraju \\
    % & & & & \\
    % & & & & \\
    \hline
  \end{tabular}
\end{center}

\noi
\StrWg{11}{17} \\
\Jest że~podmiotu w~nie większym stopniu nie~można sprowadzić
do~przedmiotu niż~przedmiotu do~podmiotu \\
\Pow że~podmiot nie może być w~większym stopniu sprowadzony
do~przedmiotu niż~przedmiot do~podmiotu \\
\StrWg{32}{3} \\
\Jest Każdy jest zniszczalny, jednakże, jak~się wydaje, istnieje ich
nieograniczone liczba w~tym samy czasie, światów zaczynających istnieć
dzięki wiecznemu ruchowi. \\
\Pow Każdy jest zniszczalny, jednakże, wydaje~się, że~nieskończona ich
liczba istnieje w~tej samej chwili, światów powstających dzięki
wiecznemu ruchowi. \\
\StrWg{32}{25} \\
\Jest poprzedzając koncepcję \\
\Pow dochodząc do koncepcji \\
\StrWd{61}{9} \\
\Jest Nie~mógł raczej być utożsamiany z~Jednym, nie~mogło~się też
zdarzyć,
by~ktoś to czynił zbyt dosłownie. \\
\Pow Nie może być utożsamiony z~Jednym, nie mogło~się też zdarzyć, by
ktoś robił to dosłownie. \\

\vspace{\spaceTwo}





% ####################
\Work{
  Jacques Maritain \\
  ,,Trzej reformatorzy. Luter, Kartezjusz, Rousseau'',
  \cite{Maritain05} }


\CenterTB{Błędy}
\begin{center}
  \begin{tabular}{|c|c|c|c|c|}
    \hline
    & \multicolumn{2}{c|}{} & & \\
    Strona & \multicolumn{2}{c|}{Wiersz}& Jest & Powinno być \\ \cline{2-3}
    & Od góry & Od dołu &  &  \\ \hline
    18 & 13 & & JózefTischner & Józef Tischner \\
    20 & & 9 & dobra & dobra'' \\
    29 & & 4 & deorsum'' & deorsum \\
    39 & 8 & & stanie & stanu \\
    % & & & & \\
    % & & & & \\
    % & & & & \\
    % & & & & \\
    % & & & & \\
    \hline
  \end{tabular}
\end{center}
\noi
\StrWg{224}{2} \\
\Jest sprawiedliwości.''., Paryż, Plon, II, 71). \\
\Pow sprawiedliwości.'', (Paryż, Plon, II, 71). \\

\vspace{\spaceTwo}





\Work{
  Jacques Maritain \\
  ,,Trzej reformatorzy'', \cite{JMTR} }


% Uwagi:\\
% \begin{itemize}
%
% \item
%
% \item
%
% \end{itemize}

\CenterTB{Błędy}
\begin{center}
  \begin{tabular}{|c|c|c|c|c|}
    \hline
    & \multicolumn{2}{c|}{} & & \\
    Strona & \multicolumn{2}{c|}{Wiersz}& Jest & Powinno być \\ \cline{2-3}
    & Od góry & Od dołu &  &  \\ \hline
    & & & & \\
    135 & 5 & & J ona & Jest ona \\
    143 & 6 & & Dostojewski Andre & Dostojewski, Andre \\
    % ,,
    150 & & 1 & Heloizy'' ,,Stąd %''
    % ,,
           & Heloizy''. ,,Stąd \\ %''
    173 & & 8 & kija obręczy & kija, obręczy \\
    & & & & \\ \hline
  \end{tabular}
\end{center}

Powinno być:
\begin{itemize}
\item[--] Str. 224. \ldots o które oskarża się z pokorą.
\end{itemize}





% ####################
\Work{
  Nakamura Hajime \\
  ,,Systemy myślenia ludów Wschodu. \\
  Indie, Chiny, Tybet, Japonia'',
  \cite{NakamuraSystemyMysleniaLudowWschodu05} }

\CenterTB{Uwagi}

\start \Str{23} Nakamura powinien tu jawnie napisać, co rozumiem przez
,,sposoby myślenia'' i~,,systemy myślenia''.

\vspace{\spaceFour}


\start \StrWd{90}{13} W~polskim tłumaczeniu użyto zwrotu ,,relacja
wyróżniającego~się i~wiedzącego'', który jest dziwny i~trudny
do~zrozumienia.

\vspace{\spaceFour}


\start \StrWd{101}{20} Ten fragment pozostawia niejasnym,
czy~działanie kogoś innego jest manifestowane, jako przedłużenie
czynności własnego ,,ja'' u~ludzi Zachodu czy~u~Indusów.

\vspace{\spaceFour}


\start \StrWg{559}{6} W~tej linii nie~powinno być wcięcia.


\CenterTB{Błędy}
\begin{center}
  \begin{tabular}{|c|c|c|c|c|}
    \hline
    & \multicolumn{2}{c|}{} & & \\
    Strona & \multicolumn{2}{c|}{Wiersz}& Jest & Powinno być \\ \cline{2-3}
    & Od góry & Od dołu &  &  \\ \hline
    69  & &  2 & jedynie rzeczownikowi & rzeczownikowi \\
    106 & 14 & & można & nie można \\
    % & & & & \\
    % & & & & \\
    559 & 20 & & naukowego]. & naukowego].] \\
    \hline
  \end{tabular}
\end{center}

\vspace{\spaceTwo}





% ########################################
\newpage
\Field{Filozofia polityczna, głównie}

\vspace{\spaceThree}
% ########################################



% ####################
\Work{
  R\'{e}mi Brague \\
  ,,Prawo Boga. Filozoficzna historia przymierza'', \cite{Brague14} }


\CenterTB{Błędy}

\noi
\tb{Grzbiet} \\
\Jest R \'{e}mi {\small Brague} \\
\Pow R\'{e}mi Brague \\
\tb{Tylna okładka} \\
\Jest R\'{e}mi {\small Brague} \\
\Pow R\'{e}mi Brague \\

\vspace{\spaceTwo}





% ####################
\Work{
  Eric Voegelin \\
  ,,Nowa nauka polityki'', \cite{VoegelinNowaNaukaPolityki92} }


\CenterTB{Uwagi}

\start Zapomniałem już na której stronie Voegelin wprowadza pojęcie
gnostyckiego snu, z~którego tłumaczeniem jest pewien problem. Po
angielsku ,,sleep'' określa stan człowieka, gdy ten śpi, lecz sny
które mogą człowieka najść określa~się słowem ,,dream''. Z~tego
względu w~języku angielskim jest jasne, czy Voegelin mówił o~,,gnostic
sleep'', czyli stanie w~którym człowiek ma radykalnie osłabiony
kontakt z~rzeczywistością, czy też o~,,gnostic dream'', czyli życiu
w~świecie dzikich fantazji. Tłumacz nie zaznaczając tej ważnej różnicy
w~tym wydaniu popełnił bardzo poważny błąd. Korzystając z~Internetu
starałem~się sprawdzić która rozumienie jest poprawne, nie mam
rozstrzygających informacji, lecz artykuł anglojęzyczny który
znalazłem, przywołując tę~myśl Voegelina, używał sformułowania
,,gnostic dream''.

\vspace{\spaceFour}


\start \Str{64} Przeczytałem tłumaczenie tu przywoływanych fragmentów
\emph{Państwa} 368 c--d i~nie znalazłem w~nich stwierdzenia,
że~\emph{polis} to wielki człowiek, była za to stwierdzenie,
że~\emph{polis} jest większe niż człowiek, więc i~sprawiedliwość jest
w~nim większa i~łatwiejsza do zauważenia. W~momencie gdy Glaukon mówi,
że~w~pierwotnym państwie jest jedzenie który i~świnie jeść by~mogły,
Sokrates powiada, że~to jest państwo jakby zdrowe, musimy więc teraz
rozważyć państwo w~którym jest bardziej dostatnie, wtedy mówi, że~jest
ono jakby w~gorączce. Są~więc obecne metafory \emph{polis} jako czegoś
żywego, ale~nie wiem czy można~się posunąć do stwierdzenia, że~jest
ono wielki człowiekiem, może jednak coś mi umyka. Opowieść o~państwie
jako o~jednym ciele jest jednak znana już w~starożytności, choćby
w~postaci bajki
\href{https://en.wikipedia.org/wiki/Agrippa_Menenius_Lanatus_(consul_503_BC)}
{Menenius Agrypy}.

\vspace{\spaceFour}


\start \StrWd{144}{} Zdanie ,,W~\emph{Prawach} Platon odsunął prawdę
duszy na~odległość jej objawienia w~\emph{Państwie}.'' jest zupełnie
niezrozumiałe, to musi byś jakiś błąd w~tłumaczeniu.



\CenterTB{Błędy}
\begin{center}
  \begin{tabular}{|c|c|c|c|c|}
    \hline
    & \multicolumn{2}{c|}{} & & \\
    Strona & \multicolumn{2}{c|}{Wiersz}& Jest & Powinno być \\ \cline{2-3}
    & Od góry & Od dołu &  &  \\ \hline
    29 & 13 & & warunkowa & warunkową \\
    35 & & 9 & ,, Filozof % ''
           & ,,Filozof \\ % ''
    45 & 10 & & sposob & sposób \\
    64 & 15 & & do do & do \\
    70 & 5 & & ton & Platon \\
    113 & 5 & & pierwszy & drugi \\
    116 & & 11 & w znaczenie & znaczenie \\
    132 & 15 & & wspierać & wspierać się \\
    141 & & 16 & f formami & z~formami \\
    153 & & 3 & wdzięczną & niewdzięczną \\
    166 & & 15 & ładząca & zaprowadzająca ład \\
    % & & & & \\
    \hline
  \end{tabular}
\end{center}

\vspace{\spaceTwo}





% ####################
\Work{
  Eric Voegelin \\
  ,,Izrael i~Objawienie'', \cite{Voegelin14} }


\CenterTB{Uwagi}

\noi \tb{Konkretne strony}

\vspace{\spaceFour}

\start \StrWd{28}{2} Mowa jest tu o~symbolizacja mikroantropicznej,
wydaje~się jednak, że~Voegelinowi chodziło o~symbolizację
makroantropiczną.

\vspace{\spaceFour}


\start \Str{54} W~przytoczonej tu~inskrypcji jest podane, że~,,Enlil
zwrócił oczy kraju [\emph{kalama}] na~siebie'', podczas gdy na dole
strony Voegelin pisze, że~oczy całego kraju Sumerów zostały skierowane
na~Lugalzagesi. Czy jest to błąd tłumaczenie, czy~też Voegelin
pozwolił sobie na~taką interpretację tego fragmentu? Jeśli to drugie,
to~należy zauważyć, iż~ta interpretacja jest dość odległa od~tekstu,
choć nie oznacza to, że~jest niepoprawna. Ja~w~każdym razie bym jej
w~takiej formie nie przyjął.

\vspace{\spaceFour}


\start \StrWg{57}{2} Jest tu mowa, że~przed stworzeniem ,,niebiańskiej
ziemi'' została stworzona ,,ziemska ziemia'', jednak kontekst
sugeruje, że~kolejność powinna być odwrotna.

\vspace{\spaceFour}


\start \Str{127} Na tej stronie pierwszy raz pojawia~się wspomniana
postać~N, ale~jej imię jest na~przemian pisane ,,N.'' albo~,,N'' i~nie
wiadomo która wersja jest poprawna.

\vspace{\spaceFour}


\start \Str{132} Cytowany tu tekst o~N przychodzącym z~Wyspy Sławy,
wykazuje duże rozbieżność z~brzmieniem, do~którego odwołuje~się
w~swojej analizie Voegelin. Na przykład w~tekście cytowany jest Wyspa
Sławy, a~Voegelin pisze o~Wyspie Płomieni.

\vspace{\spaceFour}


\start \Str{147} Na~tej stronie cytowany jest fragment~V omawianego
tekstu, ale nigdzie nie jest chyba podane, jak ten fragment brzmi.

\vspace{\spaceFour}


\start \Str{180} Ta~strona wprowadziła trochę zamieszania do~mojego
rozumienia tekstu. Przywoływane~są \emph{Hymny do~Amona}, choć wydaje
mi~się do~tej pory w~kontekście egipskim była mowa o~\emph{Hymnach
  do~Atona} (str.~163) i~\emph{Hymnach Echnatona} (str.~171). Poza
tym, są to zapewne dwie różne nazwy tego samego zbioru tekstów.

Do~tego, tekst zdaje~się mówić, że~nowa forma egipskiej religijność,
jakoś związana z~\emph{Hymnami do~Amona}, była monoteistyczna, podczas
gdy na~stronie~173 Voegelin stwierdza, że~nawet stworzenie Atona,
nadal mieściło~się w~obrębie mitu politeistycznego.

\vspace{\spaceFour}


\start \StrWd{197}{10} Zdanie ,,Przestaniemy ufać tablicy~III, lecz
odsuniemy ją na~bok'' brzmi nienaturalnie, to musi być jakaś pomyłka
tłumacza. Możliwe, że~miało być ,,Nie~przestaniemy ufać tablicy~III''.

\vspace{\spaceFour}


\start \StrWd{213}{3} W~wersie tym jest mowa o~autorach badań, lecz
chodzi raczej o~wynik pracy na Pismem~Św. autorów pracujących po
niewoli babilońskiej.

\vspace{\spaceFour}


\start \StrWd{229}{6} Użycie w~tekście polskim angielskiego słowa
,,patchwork'' nie pasuje stylu tłumaczenia. Lepiej byłoby znaleźć
polski odpowiednik tego zwrotu.

\vspace{\spaceFour}


\start \StrWg{241}{12} Rozdział 5~jest wyjątkowo krótki jak na tę
książkę i~nie dzieli się nad podrozdziały, więc odniesienie do
podrozdziału 5.2~jest błędne. Nie umiem jednak ustalić poprawnego
miejsca o~które chodzi Voegelinowi.

\vspace{\spaceFour}


\start \StrWg{312}{5} Po tej linie w~tekście powinie znajdować~się
odstęp.

\vspace{\spaceFour}


\start \StrWg{370}{12} Napisane tu jest, że~po teopolis doszło do
wycofania~się porządku w~formie kosmologicznej, jednak na podstawie
tego co Voegelin pisał wcześniej logiczniejsza byłoby inne
stwierdzenie. Mianowicie, że~po okresie teopolis wraz z~ustanowieniem
królestwa wkracza do~Izraela porządek w~formie kosmologicznej.


\CenterTB{Błędy}
\begin{center}
  \begin{tabular}{|c|c|c|c|c|}
    \hline
    & \multicolumn{2}{c|}{} & & \\
    Strona & \multicolumn{2}{c|}{Wiersz}& Jest & Powinno być \\ \cline{2-3}
    & Od góry & Od dołu &  &  \\ \hline
    52 & 13 & & 2923 & 2123 \\
    54 & & 3 & Lugalzaggesi & Lugalzagesi \\
    56 & 13 & & przepływ & na~przepływ \\
    98 & & 4 & forma'' & ,,forma'' \\
    139 & & 7 & agnostyczną metafizykę & agnostycznej metafizyce \\
    165 & & 11 & sa & są \\
    202 & 12 & & mogła & nie mogła \\
    207 & 8 & & Izraela & izraelskiego \\
    278 & & 6 & siedemnastym & dwudziestym pierwszym \\
    291 & 12 & & 13 & 14 \\
    303 & & 8 & wydarł & wydarłem \\
    314 & 6 & & 11.13 & 11, 13 \\
    316 & & 4 & miał zaś & zaś \\
    340 & & 6 & 175 & 1175 \\
    341 & 1 & & wyjaśnić,przyjmując & wyjaśnić, przyjmując \\
    357 & & 1 & 8.17 & 8, 17 \\
    365 & & 8 & nie będziecie & będziecie \\
    375 & 10 & & [zbójeckie]{ } wyprawy & [zbójeckie] wyprawy \\
    377 & & 14 & zarzadzanie & zarządzanie \\
    % & & & & \\
    % & & & & \\
    % & & & & \\
    % & & & & \\
    \hline
  \end{tabular}
\end{center}

\vspace{\spaceTwo}





% #####################################################################
% #####################################################################
\bibliographystyle{alpha} \bibliography{Bibliography}{}



\end{document}
